\documentclass{article}

\usepackage{amsmath}
\usepackage{amsfonts} % For math fonts.
\usepackage{amssymb} % For other math symbols not covered by amsmath.
\usepackage[pdftex]{graphicx} % For pictures, use \includegraphics[scale=decimal]{pic.png}; must be a .png file type.
\usepackage{multicol}
\usepackage{textcomp}
\usepackage[colorlinks = true, urlcolor = blue]{hyperref}
\usepackage{enumitem}
\usepackage{graphbox} 
\usepackage{subfig}
\usepackage{multicol}

\newcommand{\tab}{\hspace*{0.25in}}

\usepackage{tikz}
\usetikzlibrary{positioning, calc}
\usetikzlibrary{shapes.geometric,angles,quotes}
\usepackage{tikz-3dplot}

\newcommand{\csq}[1]{\reflectbox{''}#1''}  %This produces CS style quotes.



\usepackage{fullpage}
\usepackage{listings}
\lstset
{ %Formatting for code in appendix
    language=Python,
    basicstyle=\footnotesize,
    numbers=left,
    stepnumber=1,
    showstringspaces=false,
    tabsize=2,
    breaklines=true,
    breakatwhitespace=false,
}


\begin{document}


\begin{flushright}
strings\end{flushright}

\vspace*{-1.5em}
\noindent\makebox[\linewidth]{\rule{\paperwidth}{0.4pt}}


\vspace*{2em}

\begin{enumerate}


%standard 6.1


%start_of_questions


%new_question
%%%%%%%%%%%%%%%%%%%%%
	% Problem 1
	% Difficulty: 1
%%%%%%%%%%%%%%%%%%%%%
	\item 
		Zyra the code mage has hidden a mysterious cipher in reversed messages. 
		You must help Zyra uncover the secrets of the digital realm. 
		Create a function called $reverse\_string$ that takes the variable $word$ 
		(a string) and returns the word in reversed order. \\
		For this problem, you must use iteration (a loop) not slicing.

	\textbf{Examples:}
		\begin{itemize}
			\item reverse\_string(\csq{programming}) $\rightarrow$ \csq{gnimmargorp}
			\item reverse\_string(\csq{python}) $\rightarrow$ \csq{nohtyp}
			\item reverse\_string(\csq{hello}) $\rightarrow$ \csq{olleh}
		\end{itemize}


%new_question
%%%%%%%%%%%%%%%%%%%%%
	% Problem 2
	% Difficulty: 1
%%%%%%%%%%%%%%%%%%%%%
	\item 
		The \textbf{normal human body temperature} is 98.6F in Fahrenheit and 37C in Celsuis. 
		Create a function that determines if the \textit{temp} is considered a fever(anove normal 
		body temperature) or not.
		\textit{temp} will be measured in Fahrenheit and Celsuis.\\
		Notice: The F or C will always be the last character in the string.

	\textbf{Examples:}
	\begin{itemize}
		\item is\_fever(\csq{99F}) $\rightarrow$ True, 
		\item is\_fever(\csq{37C}) $\rightarrow$ False, 
		\item is\_fever(\csq{98F}) $\rightarrow$ False, 
	\end{itemize}

%new_question
%%%%%%%%%%%%%%%%%%%%%
	% Problem 3
	% Difficulty: 1
%%%%%%%%%%%%%%%%%%%%%
	\item 
		%https://edabit.com/challenge/dBqLSk6qvudNdZrSx
		The \textbf{boiling point} of water is 212F in Fahrenheit and 100C in Celsuis. 
		Create a function that determines if the \textit{temp} is considered boiling or not.
		\textit{temp} will be measured in Fahrenheit and Celsuis.\\
		Notice: The F or C will always be the last character in the string.

	\textbf{Examples:}
	\begin{itemize}
		\item is\_boiling(\csq{212F}) $\rightarrow$ True, 
		\item is\_boiling(\csq{100C}) $\rightarrow$ True, 
		\item is\_boiling(\csq{0F}) $\rightarrow$ False, 
	\end{itemize}


%new_question
%%%%%%%%%%%%%%%%%%%%%
	% Problem 4
	% Difficulty: 1
%%%%%%%%%%%%%%%%%%%%%
	\item 
		%https://edabit.com/challenge/nfWirHJzNRBMAp9Df
		The \textbf{hamming distance} is the number of characters that differ between two strings.\\
		To illustrate, \\ \hspace*{1em}
		\begin{tabular}{lll}
			str1 = &\csq{abcbba}\\
			str2 = &\csq{abcbda}
		\end{tabular}\\
		The hamming distance is 1 since the only difference is the $5^{th}$ character. \\ 
		That is, \csq{b} in str1 vs. \csq{d} in str2.
		
		Your task: create a function named hamming distance that takes two strings as arguments, 
		and returns the hamming distance between the two strings.

	\textbf{Examples:}
		\begin{itemize}
			\item hamming\_distance(\csq{abcde}, \csq{bcdef}) $\rightarrow$ 5, 
				since all 5 letters are different.
			\item hamming\_distance(\csq{abcdef}, \csq{abcdef}) $\rightarrow$ 0, 
				since all 6 letters are the same.
			\item hamming\_distance(\csq{strong}, \csq{strung}) $\rightarrow$ 1,
				since there is only 1 character that is different.
		\end{itemize}

%new_question
%%%%%%%%%%%%%%%%%%%%%
	% Problem 5
	% Difficulty: 1
%%%%%%%%%%%%%%%%%%%%%
	\item 
		%https://edabit.com/challenge/vTGXrd5ntYRk3t6Ma
		An \textbf{Isogram} is a word that has no duplicate letters. Create a function that 
		takes a string and returns either True or False depending on whether or not it is 
		an \csq{isogram}. You may assume words will only have lower case letters.
		
	\textbf{Examples:}
		\begin{itemize}
			\item is\_isogram(\csq{algorithm}) $\rightarrow$ True
			\item is\_isogram(\csq{password}) $\rightarrow$ False (multiple of s)
			\item is\_isogram(\csq{consecutive}) $\rightarrow$ False (multiple of c)
			\item is\_isogram(\csq{python}) $\rightarrow$ True
		\end{itemize}

%new_question
%%%%%%%%%%%%%%%%%%%%%
	% Problem 6
	% Difficulty: 1
%%%%%%%%%%%%%%%%%%%%%
	\item 
		%https://edabit.com/challenge/Mv5qSgZKTLrLt9zzW
		A fruit juice company tags their fruit juices by concatenating the first 
		\textbf{three letters} of the words in a flavor's name, with its capacity. 
		Create a function that creates product IDs for different fruit juices.
		Notice that the first input is a string and the second is an integer.
		
	\textbf{Examples:}
		\begin{itemize}
			\item get\_drink\_ID(\csq{apple}, 500) $\rightarrow$ \csq{app500}
			\item get\_drink\_ID(\csq{pineapple}, 45) $\rightarrow$ \csq{pin45}
			\item get\_drink\_ID(\csq{watermelon}, 750) $\rightarrow$ \csq{wat750}
		\end{itemize}





%new_question
%%%%%%%%%%%%%%%%%%%%%
	% Problem 7
	% Difficulty: 1
%%%%%%%%%%%%%%%%%%%%%
	\item 
		Professor Dumbledore seeks to decipher powerful encoded spells in the Hogwarts Library, 
		their secrets revealed by the first letter of each word. Create a function called 
		$first\_letters$ that takes the variable $sentence$ (a string) and returns
		a string made up of the first letters of each word in the sentence. 

		\textbf{Examples:}
		\begin{itemize}
			\item first\_letters(\csq{wingardium leviosa makes objects float}) 
				$\rightarrow$ \csq{wlmof}
			\item first\_letters(\csq{expecto patronum repels dementors}) $\rightarrow$ \csq{eprd}
			\item first\_letters(\csq{the magic is within you}) $\rightarrow$ \csq{tmiwy}
		\end{itemize}

%new_question
%%%%%%%%%%%%%%%%%%%%%
	% Problem 8
	% Difficulty: 1
%%%%%%%%%%%%%%%%%%%%%
	\item 
		Severus Snape seeks to harness powerful spells in the Hogwarts Library, you must 
		encode them by using the last letter of each word. Create a function called 
		$last\_letters$ that takes the variable $sentence$ (a string) and returns
		a string made up of the last letters of each word in the sentence. 

		\textbf{Examples:}
		\begin{itemize}
			\item last\_letters(\csq{wingardium leviosa makes objects float}) 
				$\rightarrow$ \csq{masst}
			\item last\_letters(\csq{expecto patronum repels dementors}) $\rightarrow$ \csq{omss}
			\item last\_letters(\csq{the magic is within you}) $\rightarrow$ \csq{ecsnu}
		\end{itemize}



%new_question
%%%%%%%%%%%%%%%%%%%%%
	% Problem 9
	% Difficulty: 1
%%%%%%%%%%%%%%%%%%%%%
	\item
		Write a function called \textit{flip\_flop} that takes a string as an argument 
		and returns  a new word made up of the second half of the word first combined 
		with the first half of the word second.

		\textbf{Examples:}
		\begin{itemize}
			\item \textit{flip\_flop}(\csq{abcd}) $\rightarrow$ \csq{cdab} 
				(that is, \csq{cd} then \csq{ab} \dots even length)
			\item \textit{flip\_flop}(\csq{grapes}) $\rightarrow$ \csq{pesgra} 
				(that is, \csq{pes} then \csq{gra} \dots even length)
			\item \textit{flip\_flop}(\csq{abcde})$\rightarrow$ \csq{decab}
				(that is, \csq{de} then \csq{c} then \csq{ab} \dots odd length)
			\item \textit{flip\_flop}(\csq{cranberries})$\rightarrow$ \csq{rriesecranb}
				(that is, \csq{rries} then \csq{e} then \csq{cranb} \dots odd length)

		\end{itemize}

%end_of_questions
%make sure to leave at least one blank line below




\end{enumerate}
\end{document}

%https://leetcode.com/problems/roman-to-integer/description/?envType=problem-list-v2&envId=hash-table








%new_question
%%%%%%%%%%%%%%%%%%%%%
	% Problem 1
	% Difficulty: 1
%%%%%%%%%%%%%%%%%%%%%
	%paper-based
	\item
		Write a function named $isPalindrome$ to determine if a string is a palindrome. It should take a single string as an argument
		and return True if it is a palindrome and False otherwise. You may assume each input string has no punctuation, capitalization or spacing.
		Do not use the built-in function reverse().
		Clarification: A word is a palindrome if it reads the same forwards as it does backwards.
		For example, "bob" is a palindrome, and "neveroddoreven" is a palindrome.
		However, "helloworld" is not a palindrome because "helloworld" is different than "dlrowolleh"


%new_question
%%%%%%%%%%%%%%%%%%%%%
	% Problem 1
	% Difficulty: 1
%%%%%%%%%%%%%%%%%%%%%
	%paper-based
	\item
		Write a function that takes a string and a letter as arguments. The function should return the amount of times
		the letter appears in the string. For example, if the string was "frantic running" and the letter was "n", the function should
		return 4 as "n" appears 4 times in the string.
		Do not use the built-in function count().
%new_question
%%%%%%%%%%%%%%%%%%%%%
	% Problem 1
	% Difficulty: 1
%%%%%%%%%%%%%%%%%%%%%
	%paper-based
	\item
		Write a function that takes a string as an argument, loops through the string, and prints every other letter starting
		with the second letter. For example, if the word was "bellyflop", the function should print "elfo". If the word was
		"backflip", it should print "aklp".


%new_question
%%%%%%%%%%%%%%%%%%%%%
	% Problem 1
	% Difficulty: 1
%%%%%%%%%%%%%%%%%%%%%
	%paper-based
	\item
		In British English, there is a word $flavour$. The American spelling is "flavor". There are 
		many such examples, colour vs color, humour vs humor,
		etc. Write a function that takes a word with the British spelling and returns it with the
		American spelling. Do not use the built-in function replace().


%new_question
%%%%%%%%%%%%%%%%%%%%%
	% Problem 1
	% Difficulty: 1
%%%%%%%%%%%%%%%%%%%%%
	%paper-based
	\item 
		Write a function that takes a string as an argument and returns the length of the string.
		\textbf{Examples:}
		\begin{itemize}
			\item \csq{apple} $\rightarrow$ 5
			\item \csq{hello world} $\rightarrow$ 11
			\item \csq{MSU} $\rightarrow$ 3
		\end{itemize}


%new_question
%%%%%%%%%%%%%%%%%%%%%
	% Problem 1
	% Difficulty: 1
%%%%%%%%%%%%%%%%%%%%%
	%paper-based
	\item 
		Write a function that takes a string as an argument and returns the length of the string.
		\textbf{Examples:}
		\begin{itemize}
			\item \csq{apple} $\rightarrow$ 5
			\item \csq{hello world} $\rightarrow$ 11
			\item \csq{MSU} $\rightarrow$ 3
		\end{itemize}

%new_question
%%%%%%%%%%%%%%%%%%%%%
	% Problem 1
	% Difficulty: 1
%%%%%%%%%%%%%%%%%%%%%
	%paper-based
	\item 
		Write a function named $e\_counter$ that takes a string as an argument and returns the number of times "e" appears in the string.
		Do NOT use the built-in function count().
		\textbf{Examples:}
		\begin{itemize}
			\item \textit{e\_counter}(\csq{green}) $\rightarrow$ 2
			\item \textit{e\_counter}(\csq{wheels need speed}) $\rightarrow$ 6
			\item \textit{e\_counter}(\csq{banana}) $\rightarrow$ 0
		\end{itemize}

%new_question
%%%%%%%%%%%%%%%%%%%%%
	% Problem 1
	% Difficulty: 1
%%%%%%%%%%%%%%%%%%%%%
	\item
		When driving in a car and approaching a traffic control light, \textit{green} means go, 
		\textit{yellow} means yield, and \textit{red} means stop.  Assuming there is a variable 
		named \textit{light\_color}, write a program that prints either the word \textit{go}, 
		\textit{yield}, or \textit{red} depending of the value of \textit{light\_color}.  
		Let the user input the value of \textit{light\_color}.

%new_question
%%%%%%%%%%%%%%%%%%%%%
	% Problem 1
	% Difficulty: 1
%%%%%%%%%%%%%%%%%%%%%
	\item 
		%https://edabit.com/challenge/Ay9wPrqRJnBmvbFmi
		Ask the user for two integers, \textit{larger} and \textit{smaller}.  Determine (and output) 
		how many times larger can be halved while still be greater than smaller.
		
		Examples:
		\begin{itemize}
			\item if \textit{larger} = 1324 and  \textit{smaller} = 98, the result should be 3 since
				1324 $\rightarrow$ 662 $\rightarrow$ 331 $\rightarrow$ 165.5
			\item if \textit{larger} = 624 and  \textit{smaller} = 8, the result should be 6 since\\
				\tab 624 $\rightarrow$ 312 $\rightarrow$ 156 $\rightarrow$ 78 $\rightarrow$ 39 
				$\rightarrow$ 19.5 $\rightarrow$ 9.75)
		\end{itemize}



%new_question
%%%%%%%%%%%%%%%%%%%%%
	% Problem 1
	% Difficulty: 1
%%%%%%%%%%%%%%%%%%%%%
	\item 
		%https://edabit.com/challenge/68KgdtdwabrXydZFM
		Create a function that takes an integer named \textit{number} 
		%and a character named \textit{gender} (\csq{m} for male, \csq{f} for female), 
		and returns the name of an ancestor or\\ descendant based on the table below.
		Your code should work for any integer.
		\begin{flushright}
	    	\begin{tabular}{|c|c|c|} \hline
				\textbf{Generation} &  \\ \hline
	    	    -3 	& great grandparent \\
	    	    -2 	& grandparent \\
	    	    -1 	& parent\\
	    	   	0 	& me! \\
	    	 	1 	& child \\
	    	 	2 	& grandchild \\
		   	  	3 	& great grandchild\\ 
		   	  	\vdots & \\
		   	  	6 & great great great great grandchild \\ \hline
		    \end{tabular}
		\end{flushright}


%new_question
%%%%%%%%%%%%%%%%%%%%%
	% Problem 1
	% Difficulty: 1
%%%%%%%%%%%%%%%%%%%%%
	%new
	\item 
		In text analysis, finding the shortest word in a sentence is a common task. 
		Create a function called $shortest\_word$ that 
		takes the variable $sentence$ (a string) and returns the shortest word in the sentence. 

	\textbf{Examples:}
		\begin{itemize}
			\item shortest\_word(\csq{the shining moon}) $\rightarrow$ \csq{the}
			\item shortest\_word(\csq{jumped on the trampoline}) $\rightarrow$ \csq{on}
			\item shortest\_word(\csq{hello planet}) $\rightarrow$ \csq{hello}
		\end{itemize}


%new_question
%%%%%%%%%%%%%%%%%%%%%
	% Problem 7
	% Difficulty: 2
%%%%%%%%%%%%%%%%%%%%%
	\item 
		Reversing the order of words in a sentence can be helpful in text processing. 
		Create a function called $reverse\_words$ that takes the variable $sentence$ (a string) 
		and returns the sentence with its words in reversed order. 

		\textbf{Examples:}
		\begin{itemize}
			\item reverse\_words(\csq{programming useful tools}) 
				$\rightarrow$ \csq{tools useful programming}
			\item reverse\_words(\csq{python is a great language}) 
				$\rightarrow$ \csq{language great is a python}
			\item reverse\_words(\csq{hello world}) $\rightarrow$ \csq{world hello}
		\end{itemize}


%new_question
%%%%%%%%%%%%%%%%%%%%%
	% Problem 1
	% Difficulty: 1
%%%%%%%%%%%%%%%%%%%%%
	%new
	\item 
		A Sith infiltrator has scrambled the ancient Jedi texts, swapping the first half of each 
		passage with the second. You must help master Yoda restore the order of the archives. 
		Create a function called $swap\_sentence\_halves$ that takes the variable $sentence$ 
		(a string) and returns the sentence with the first half swapped with the second half. 
		If the sentence contains an odd number of words, the extra word should remain in the 
		second half. 

		\textbf{Examples:}
		\begin{itemize}
			\item swap\_sentence\_halves(\csq{hello there general kenobi}) 
				$\rightarrow$ \csq{general kenobi hello there}
			\item swap\_sentence\_halves(\csq{the force will be with you always}) 
				$\rightarrow$ \csq{be with you always the force will}
			\item swap\_sentence\_halves(\csq{do or not do there is no try}) 
				$\rightarrow$ \csq{there is no try do or not do}
		\end{itemize}

%maybe
%https://edabit.com/challenge/eADRy5SA5QbasA3Qt
