\documentclass{article}

\usepackage{amsmath}
\usepackage{amsfonts} % For math fonts.
\usepackage{amssymb} % For other math symbols not covered by amsmath.
\usepackage[pdftex]{graphicx} % For pictures, use \includegraphics[scale=decimal]{pic.png}; must be a .png file type.
\usepackage{multicol}
\usepackage{textcomp}
\usepackage[colorlinks = true, urlcolor = blue]{hyperref}
\usepackage{enumitem}
\usepackage{graphbox} 
\usepackage{subfig}
\usepackage{multicol}
\usepackage{nopageno}
\usepackage{bm}


\usepackage{tikz}
\usetikzlibrary{positioning, calc}
\usetikzlibrary{shapes.geometric,angles,quotes}
\usepackage{tikz-3dplot}


%page formatting
\usepackage{fullpage}
\setlength{\parindent}{0pt}


\newcommand{\tab}{\hspace*{0.25in}}
\newcommand{\csq}[1]{\reflectbox{''}#1''}  %This produces CS style quotes.
\newcommand{\csqt}[1]{\text{\reflectbox{''}#1''}}  %This produces CS style quotes as text.


\usepackage{listings}
\lstset
{ %Formatting for code in appendix
    language=Python,
    basicstyle=\footnotesize,
    numbers=left,
    stepnumber=1,
    showstringspaces=false,
    tabsize=2,
    breaklines=true,
    breakatwhitespace=false,
}


\begin{document}



%split_point

%\end{document}
Lone Star \hfill Classes Part 1 quiz\\
section 4\\
\begin{enumerate}
\item (11.1) Create a \textit{Book} class.\\	
	\begin{minipage}{.6\textwidth}
		A \textit{Book} has
		\begin{itemize}
			\item title 
			\item author
			\item page\_count	
		\end{itemize}
	\end{minipage}
	%
	\begin{minipage}{.4\textwidth}
		This class ``looks'' like 
				
		\vspace*{1em}
		\begin{tabular}{|l|}
			\hline Book\\ \hline
			title\\ author\\ page\_count\\ \hline
			\\  \hline
		\end{tabular}
	\end{minipage}

	\vspace*{2ex}
	Create a constructor method that initializes all instance variables.\\
	You should write getters and setters for each of the instance variables.\\
	Instantiate an instance of the class. You may pass any initial values of your choosing.


\item (11.2) Create a \textit{Vehicle} class.\\
	\begin{minipage}{.6\textwidth}	
		A \textit{Vehicle} has
		\begin{itemize}
			\item make 
			\item model
			\item year	
		\end{itemize}
		
		A \textit{Vehicle} can do
		\begin{itemize}
			\item \textit{print\_vehicle\_type}
		\end{itemize}
	\end{minipage}
	%
	\begin{minipage}{.4\textwidth}
		This class ``looks'' like 
				
		\vspace*{1em}
		\begin{tabular}{|l|}
			\hline Vehicle\\ \hline
			make\\ model\\ year\\ \hline
			print\_vehicle\_type \\  \hline
		\end{tabular}
	\end{minipage}

	\vspace*{2ex}
	Create a constructor method that initializes all instance variables.\\
	You should write getters and setters for each of the instance variables.\\
	Instantiate an instance of the class. You may pass any initial values of your choosing.

	Write a method called \textit{print\_vehicle\_type}, which prints in the form ``[year] [make] [model]''\\
	example. ``2021 Toyota Camry''.\\

	%Create a \_\_str\_\_ method that returns a string in the format:\\
	%"[year] [make] [model]"\\
	%\tab \tab eg. "2021 Toyota Camry".\\


\item (11.3) Create a \textit{Student} class.\\
	\begin{minipage}{.6\textwidth}		
		A \textit{Student} has
		\begin{itemize}
			\item A name
			\item A major
			\item A GPA	
		\end{itemize}
	
		A \textit{Student} can do
		\begin{itemize}
			\item introduce themselves
			\item study for exam
		\end{itemize}
	\end{minipage} 
	%
	\begin{minipage}{.4\textwidth}
		This class \csq{looks} like
		 
		\vspace*{1em}
		\begin{tabular}{|l|}
			\hline Student\\ \hline
			name\\ major\\ GPA\\ \hline
			introduce\\ study\_for\_exam \\  \hline
		\end{tabular}
	\end{minipage}

	\vspace*{2ex}
	You should write getters and setters for each of the instance variables.\

	An introduction should be of the form: \underline{Hi, I'm  \textit{name}.  
	I'm studying \textit{major}.}\\
	\tab \tab eg. Hi. I'm Maria. I'm studying Computer Science.\\

	Studying for an exam should increase the GPA by \underline{0.2} points. (up to a maximum of 4.0)\\  
	It should be of the form: \\
	I'm hitting the books! My GPA increased from \textit{old GPA} to \textit{new GPA}.\\
	\tab \tab eg. I'm hitting the books! My GPA increased from 3.5 to 3.7.\\




\end{enumerate}
\pagebreak
Dot Matrix \hfill Classes Part 1 quiz\\
section 5\\
\begin{enumerate}
\item (11.1) Create an \textit{Product} class.\\
	\begin{minipage}{.6\textwidth}
		A \textit{product} has
		\begin{itemize}
			\item A name
			\item A price
			\item A quantity	
		\end{itemize}
	\end{minipage} 
	%
	\begin{minipage}{.4\textwidth}
		This class ``looks'' like 
				
		\vspace*{1em}
		\begin{tabular}{|l|}
			\hline Product\\ \hline
			name\\ price\\ quantity\\ \hline
			\\  \hline
		\end{tabular}
	\end{minipage}

	\vspace*{2ex}
	Create a constructor method that initializes all instance variables.\\
	You should write getters and setters for each of the instance variables.\\
	Instantiate an instance of the class. You may pass any initial values of your choosing.

\item (11.2) Create a \textit{Point} class.\\
	\begin{minipage}{.6\textwidth}		
		A \textit{Point} has
		\begin{itemize}
			\item x\_coordinate 
			\item y\_coordinate 
		\end{itemize}

		A \textit{Point} can do
		\begin{itemize}
			\item \textit{print\_info}
		\end{itemize}
	\end{minipage}
	%
	\begin{minipage}{.4\textwidth}
		This class ``looks'' like 
				
		\vspace*{1em}
		\begin{tabular}{|l|}
			\hline Course\\ \hline
			x\_coordinate\\ y\_coordinate\\ \hline
			print\_info\\  \hline
		\end{tabular}
	\end{minipage}

	\vspace*{2ex}
	Create a constructor method that initializes all instance variables.\\
	You should write getters and setters for each of the instance variables.\\
	Instantiate an instance of the class. You may pass any initial values of your choosing.

	Write a method called \textit{print\_info}, which prints in the form \\
		\tab \tab \tab ``(x,y)=([x\_coordinate], [y\_coordinate])''\\
	example. ``(x,y)=( 4, 5 )''.\\

	%Create a \_\_str\_\_ method that returns a string in the format:\\
	%"( [x\_coordinate], [y\_coordinate], [z\_coordinate] )"\\
	%\tab \tab eg. "( 4, 5, 6 )".\\


%end_of_questions
%make sure to leave at least one blank line below


\item (11.3) Create an \textit{Employee} class.\\
	\begin{minipage}{.6\textwidth}		
		An \textit{Employee} has
		\begin{itemize}
			\item A name
			\item A title
			\item A salary	
		\end{itemize}
	
		An \textit{Employee} can do
		\begin{itemize}
			\item a greeting
			\item request raise
		\end{itemize}
	\end{minipage} 
	%
	\begin{minipage}{.4\textwidth}
		This class ``looks'' like 
				
		\vspace*{1em}
		\begin{tabular}{|l|}
			\hline Employee\\ \hline
			name\\ title\\ salary\\ \hline
			greeting\\ request\_raise \\  \hline
		\end{tabular}
	\end{minipage}

	\vspace*{2ex}
	You should write getters and setters for each of the instance variables.\\

	A greeting should be of the form: \underline{Hello.  My name is \textit{name}.  
	I'm the \textit{title}.}\\
	\tab \tab eg. Hello.  My name is Eugene.  I'm the CEO.\\

	A raise request should request a \underline{6\%} raise.\\  It should be of the form: 
	I'm currently making \textit{salary}.  I'd like new salary of \textit{new amount}.\\
	\tab \tab eg. I'm currently making \$100.  I'd like new salary of \$106.\\



\end{enumerate}
\pagebreak
Dark Helmet \hfill Classes Part 1 quiz\\
section 4\\
\begin{enumerate}
\item (11.1) Create a \textit{Movie} class.\\
	\begin{minipage}{.6\textwidth}		
		A \textit{Movie} has
		\begin{itemize}
			\item title 
			\item director
			\item runtime\_minutes	
		\end{itemize}
	\end{minipage}
	%
	\begin{minipage}{.4\textwidth}
		This class ``looks'' like 
				
		\vspace*{1em}
		\begin{tabular}{|l|}
			\hline Movie\\ \hline
			title\\ director\\ runtime\_minutes\\ \hline
			\\  \hline
		\end{tabular}
	\end{minipage}

	\vspace*{2ex}
	Create a constructor method that initializes all instance variables.\\
	You should write getters and setters for each of the instance variables.\\
	Instantiate an instance of the class. You may pass any initial values of your choosing.


\item (11.2) Create a \textit{Course} class.\\
	\begin{minipage}{.6\textwidth}
		A \textit{Course} has
		\begin{itemize}
			\item course\_code 
			\item course\_name
			\item instructor	
		\end{itemize}

		An \textit{Course} can do
		\begin{itemize}
			\item \textit{print\_info}
		\end{itemize}	
	\end{minipage}
	%
	\begin{minipage}{.4\textwidth}
		This class ``looks'' like 
				
		\vspace*{1em}
		\begin{tabular}{|l|}
			\hline Course\\ \hline
			course\_code\\ course\_name\\ instructor\\ \hline
			print\_info\\  \hline
		\end{tabular}
	\end{minipage}

	\vspace*{2ex}
	Create a constructor method that initializes all instance variables.\\
	You should write getters and setters for each of the instance variables.\\
	Instantiate an instance of the class. You may pass any initial values of your choosing.

	Write a method called \textit{print\_info}, which prints in the form \\
		\tab \tab \tab ``[course\_code]: [course\_name] taught by [instructor]''\\
	example. ``CIS101: Introduction to programming taught by Matt''.\\

	%Create a \_\_str\_\_ method that returns a string in the format:\\
	%"[course\_code]: [course\_name] taught by [instructor], has a max capacity of [max\_capacity]"\\
	%\tab \tab eg. "CIS101: Introduction to programming taught by Dr.Smith, has a max capacity of 25".\\


\item (11.3) Create a \textit{Circle} class.\\	
	\begin{minipage}{.6\textwidth}
		A \textit{Circle} has
		\begin{itemize}
			\item radius 
		\end{itemize}

		A \textit{Circle} can do
		\begin{itemize}
			%\item calculate\_area()
			\item calculate\_circumference
		\end{itemize}
	\end{minipage}
		%
	\begin{minipage}{.4\textwidth}
		This class ``looks'' like 
				
		\vspace*{1em}
		\begin{tabular}{|l|}
			\hline Circle\\ \hline
			radius\\ \ \\  \hline
			calculate\_circumference\\ \hline
		\end{tabular}
	\end{minipage}

	\vspace*{2ex}
	Create a constructor method that initializes all instance variables.\\
	You should write getters and setters for each of the instance variables.\\
	Instantiate an instance of the class. You may pass any initial values of your choosing.	

	%The calculate\_area() method should return the area calculated as: $\pi \cdot \text{radius}^2$.\\
	The calculate\_circumference() method should return the circumference calculated as: $2 \cdot \pi \cdot \text{radius}$.\\

	%Create an \_\_str\_\_ method that returns a string in the format:\\
	%"Circle(radius=[radius], color=[color], filled=[filled])"\\
	%\tab \tab eg. "Circle(radius=5.0, color=red, filled=True)".\\


%end_of_questions
%make sure to leave at least one blank line below


\end{enumerate}
\pagebreak
President Skroob \hfill Classes Part 1 quiz\\
section 1\\
\begin{enumerate}
\item (11.1) Create an \textit{Product} class.\\
	\begin{minipage}{.6\textwidth}
		A \textit{product} has
		\begin{itemize}
			\item A name
			\item A price
			\item A quantity	
		\end{itemize}
	\end{minipage} 
	%
	\begin{minipage}{.4\textwidth}
		This class ``looks'' like 
				
		\vspace*{1em}
		\begin{tabular}{|l|}
			\hline Product\\ \hline
			name\\ price\\ quantity\\ \hline
			\\  \hline
		\end{tabular}
	\end{minipage}

	\vspace*{2ex}
	Create a constructor method that initializes all instance variables.\\
	You should write getters and setters for each of the instance variables.\\
	Instantiate an instance of the class. You may pass any initial values of your choosing.

\item (11.2) Create a \textit{Vehicle} class.\\
	\begin{minipage}{.6\textwidth}	
		A \textit{Vehicle} has
		\begin{itemize}
			\item make 
			\item model
			\item year	
		\end{itemize}
		
		A \textit{Vehicle} can do
		\begin{itemize}
			\item \textit{print\_vehicle\_type}
		\end{itemize}
	\end{minipage}
	%
	\begin{minipage}{.4\textwidth}
		This class ``looks'' like 
				
		\vspace*{1em}
		\begin{tabular}{|l|}
			\hline Vehicle\\ \hline
			make\\ model\\ year\\ \hline
			print\_vehicle\_type \\  \hline
		\end{tabular}
	\end{minipage}

	\vspace*{2ex}
	Create a constructor method that initializes all instance variables.\\
	You should write getters and setters for each of the instance variables.\\
	Instantiate an instance of the class. You may pass any initial values of your choosing.

	Write a method called \textit{print\_vehicle\_type}, which prints in the form ``[year] [make] [model]''\\
	example. ``2021 Toyota Camry''.\\

	%Create a \_\_str\_\_ method that returns a string in the format:\\
	%"[year] [make] [model]"\\
	%\tab \tab eg. "2021 Toyota Camry".\\


\item (11.3) Create a \textit{TemperatureInCelsius} class.\\
	\begin{minipage}{.6\textwidth}		
		A \textit{TemperatureInCelsius} has
		\begin{itemize}
			\item temp\_value
		\end{itemize}

		A \textit{TemperatureInCelsius} can do
		\begin{itemize}
			\item to\_fahrenheit
		\end{itemize}
	\end{minipage}
		%
	\begin{minipage}{.4\textwidth}
		This class ``looks'' like 
				
		\vspace*{1em}
		\begin{tabular}{|l|}
			\hline TemperatureInCelsius\\ \hline
			temp\_value\\ \ \\  \hline
			to\_fahrenheit\\ \ \\ \hline
		\end{tabular}
	\end{minipage}



	\vspace*{2ex}
	Clarification: temp\_value is the temperature in Celsius.\\
	Create a constructor method that initializes all instance variables.\\
	You should write getters and setters for each of the instance variables.\\
	Instantiate an instance of the class. You may pass any initial values of your choosing.
	
	The to\_fahrenheit() method should return the temperature in Fahrenheit calculated as:\\
	Fahrenheit = (Celsius * 9/5) + 32.\

	%Create an \_\_str\_\_ method that returns a string in the format:\\
	%"Temp: [celsius]C, Humidity: [humidity]\%, Pressure: [pressure]hPa"\\
	%\tab \tab eg. "Temp: 25.0C, Humidity: 65\%, Pressure: 1013hPa".\\


\end{enumerate}
\pagebreak
\end{document}