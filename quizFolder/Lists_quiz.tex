\documentclass{article}

\usepackage{amsmath}
\usepackage{amsfonts} % For math fonts.
\usepackage{amssymb} % For other math symbols not covered by amsmath.
\usepackage[pdftex]{graphicx} % For pictures, use \includegraphics[scale=decimal]{pic.png}; must be a .png file type.
\usepackage{multicol}
\usepackage{textcomp}
\usepackage[colorlinks = true, urlcolor = blue]{hyperref}
\usepackage{enumitem}
\usepackage{graphbox} 
\usepackage{subfig}
\usepackage{multicol}
\usepackage{nopageno}
\usepackage{bm}


\usepackage{tikz}
\usetikzlibrary{positioning, calc}
\usetikzlibrary{shapes.geometric,angles,quotes}
\usepackage{tikz-3dplot}


%page formatting
\usepackage{fullpage}
\setlength{\parindent}{0pt}


\newcommand{\tab}{\hspace*{0.25in}}
\newcommand{\csq}[1]{\reflectbox{''}#1''}  %This produces CS style quotes.
\newcommand{\csqt}[1]{\text{\reflectbox{''}#1''}}  %This produces CS style quotes as text.


\usepackage{listings}
\lstset
{ %Formatting for code in appendix
    language=Python,
    basicstyle=\footnotesize,
    numbers=left,
    stepnumber=1,
    showstringspaces=false,
    tabsize=2,
    breaklines=true,
    breakatwhitespace=false,
}


\begin{document}



%split_point

%\end{document}
Lone Star \hfill Lists quiz\\
section 4\\
\begin{enumerate}
\item (7.1) 
		Write a \textbf{function} that loops through and returns a list with every even number between two
		integers (inclusive). The arguments to the function will be $smaller\_num$ and 
		$larger\_num$.

		\textbf{Examples:}		
		\begin{itemize}
			\item  output\_even(37, 1050) $\rightarrow$ [38, 40, 42, \dots 1048, 1050], 
			\item  output\_even(1, 2000) $\rightarrow$ [2, 4, 6, \dots 1998, 2000], 
			\item  output\_even(50, 199) $\rightarrow$ [50, 52, 54, \dots 196, 198]
		\end{itemize}

\item (7.2) 
		Write a \textbf{function} that finds the largest even number in a list $numbers$. Return -1 if not found. 
		You may \textbf{not} use the built-in functions \textit{max}(), \textit{min}(), \textit{sort}(), or \textit{sorted}().

		\textbf{Examples:}		
		\begin{itemize}
			\item  largest\_even([3, 7, 2, 1, 7, 9, 10, 13]) $\rightarrow$ 10,
			\item  largest\_even([1, 3, 5, 7]) $\rightarrow$ -1,
			\item  largest\_even([0, 19, 18973623]) $\rightarrow$ 0
		\end{itemize}

\item (7.3)
		To train for an upcoming marathon, Johnny goes on one long-distance run each Saturday. 
		He wants to track how often the number of miles he runs exceeds the previous Saturday. 
		This is called a progress day. Write a \textbf{function} that takes in a list of miles 
		run every Saturday and returns Johnny's total number of progress days.

		\textbf{Examples:}		
		\begin{itemize}
			\item  progress\_days([3, 4, 1, 2]) $\rightarrow$ 2, 
				(Two progress days, day 2 since $(4>3)$ and day 4 since $(2>1)$)
			\item  progress\_days([10, 11, 12, 9, 10]) $\rightarrow$ 3, 
			\item  progress\_days([6, 5, 4, 3, 2, 9]) $\rightarrow$ 1, 
			\item  progress\_days([9, 9]) $\rightarrow$ 0
		\end{itemize}

\end{enumerate}
\pagebreak
Dot Matrix \hfill Lists quiz\\
section 5\\
\begin{enumerate}
\item (7.1) 
		Write a \textbf{function} that loops through a word and returns a list with every 
		other letter of the word starting with the \textbf{first} letter.
		The function will take a single argument $word$ (a string representing the word to process).

		\textbf{Examples:}		
		\begin{itemize}
			\item  skip\_letter(\csq{counterattack}) $\rightarrow$ 
				[\csq{c},\csq{u},\csq{t},\csq{r},\csq{t},\csq{a},\csq{c}]
			\item  skip\_letter(\csq{banana sunday}) $\rightarrow$
				[\csq{b},\csq{n},\csq{n},\csq{s},\csq{n},\csq{a}]
		\end{itemize}

\item (7.2) 
		%https://edabit.com/challenge/6Pf5GGG6HnzbB95gf
		Write a \textbf{function} that returns a list with the factors of a given integer. The argument of the function
		will be $num$ (integer to find factors for).

		\textbf{Examples:}		
		\begin{itemize}
			\item  find\_factors(12) $\rightarrow$ [1, 2, 3, 4, 6, 12], 
			\item  find\_factors(17) $\rightarrow$ [1, 17],
			\item  find\_factors(36) $\rightarrow$ [1, 2, 3, 4, 6, 9, 12, 18, 36]
		\end{itemize}



\item (7.3) 
		There's a great war between the even and odd numbers. Many numbers already lost 
		their lives in this war and it's your task to end this. You have to determine 
		which group sums larger: the evens or the odds. The larger group wins.

		Write a \textbf{function} that takes a list of integers named \textit{numbers}, 
		sums the even numbers and odd numbers separately, then returns which of the two
		sums is larger.

		\textbf{Examples:}		
		\begin{itemize}
			\item  war\_of\_numbers([2, 8, 7, 5]) $\rightarrow$ 
				\csq{odds}, (since 2 + 8 = 10, 7 + 5 = 12, odds is larger) 
			\item  war\_of\_numbers([12, 90, 75, 1, 1]) $\rightarrow$ \csq{evens}, 
				(12 + 90 = 102, 75 + 1 + 1 = 77, evens is larger) 
			\item  war\_of\_numbers([2, 10, 22, 243]) $\rightarrow$ \csq{odds}
		\end{itemize}



\end{enumerate}
\pagebreak
Dark Helmet \hfill Lists quiz\\
section 4\\
\begin{enumerate}
\item (7.1) 
		Write a \textbf{function} that loops through a word and returns a list with every 
		other letter of the word starting with the \textbf{second} letter.
		The function will take a single argument $word$ (a string representing the word to process).

		\textbf{Examples:}		
		\begin{itemize}
			\item  skip\_letter(\csq{counterattack}) $\rightarrow$ 
				[\csq{o},\csq{n},\csq{e},\csq{a},\csq{t},\csq{c}]
			\item  skip\_letter(\csq{banana sunday}) $\rightarrow$
				[\csq{a},\csq{a},\csq{a},\csq{s},\csq{n},\csq{a}]
		\end{itemize}


\item (7.2) 
		Write a function named \textit{add\_lists} that takes two lists $lyst1$ and $lyst2$ and adds the 
		first element in $lyst1$ with the first element in $lyst2$, the second element $lyst1$
		with the second element $lyst2$, etc. Return a new list containing the corresponding 
		sums of the list1 and list2.  You may assume both lists have the same length.
		%Return True if all element combinations add up to the same number. Otherwise, return False.

		\textbf{Examples:}		
		\begin{itemize}
			\item  add\_lists([1, 3, 3, 1], [4, 3, 6, 1]) $\rightarrow$ [5, 6, 8, 2] 
				( since 1+4=5; 3+3=6; 3+6=9; 1+1=2)
			\item  add\_lists([1, 8, 5, 0, -7], [0, -7, 4, 2, -6]) $\rightarrow$ [1, 1, 9, 2, -13]
			\item  add\_lists([1, 2], [-1, 1]) $\rightarrow$ [0, 3]
		\end{itemize}


\item (7.3)
		To train for an upcoming marathon, Samuel goes on one long-distance run each Saturday. 
		He wants to track how often the number of miles he runs fall short of the previous Saturday. 
		This is called a lag day. Write a \textbf{function} that takes in a list of miles 
		run every Saturday and returns Samuel's total number of lag days.

		\textbf{Examples:}		
		\begin{itemize}
			\item  lag\_days([5, 3, 2, 1]) $\rightarrow$ 3, 
				(3 lag days, day2 since (3$<$5), day3 since (2$<$3), and day4 since (1$<$2))
			\item  lag\_days([10, 11, 12, 9, 10]) $\rightarrow$ 1, 
			\item  lag\_days([6, 5, 4, 3, 2, 9]) $\rightarrow$ 4, 
			\item  lag\_days([9, 9]) $\rightarrow$ 0
		\end{itemize}

\end{enumerate}
\pagebreak
President Skroob \hfill Lists quiz\\
section 1\\
\begin{enumerate}
\item (7.1) 
		Write a \textbf{function} that loops through a word and returns a list with every 
		other letter of the word starting with the \textbf{first} letter.
		The function will take a single argument $word$ (a string representing the word to process).

		\textbf{Examples:}		
		\begin{itemize}
			\item  skip\_letter(\csq{counterattack}) $\rightarrow$ 
				[\csq{c},\csq{u},\csq{t},\csq{r},\csq{t},\csq{a},\csq{c}]
			\item  skip\_letter(\csq{banana sunday}) $\rightarrow$
				[\csq{b},\csq{n},\csq{n},\csq{s},\csq{n},\csq{a}]
		\end{itemize}

\item (7.2)
		Write a \textbf{function} that takes 3 numbers as arguments, $num\_1$ (first number), 
		$num\_2$ (second number), and $num\_3$ (third number). 
		Return a list of the integers in ascending order. 
		You may \textbf{not} use the built-in functions \textit{max}(), \textit{min}(), 
		\textit{sort}(), or \textit{sorted}().
		
	\textbf{Examples:}
	\begin{itemize}
		\item  ascending\_order(2, 3, 1) $\rightarrow$ [1, 2, 3], 
		\item  ascending\_order(10, 1, 25) $\rightarrow$ [1, 10, 25], 
		\item  ascending\_order(2, 45, 4) $\rightarrow$ [2, 4, 45] 
	\end{itemize}


\item (7.3)
		To train for an upcoming marathon, Samuel goes on one long-distance run each Saturday. 
		He wants to track how often the number of miles he runs fall short of the previous Saturday. 
		This is called a lag day. Write a \textbf{function} that takes in a list of miles 
		run every Saturday and returns Samuel's total number of lag days.

		\textbf{Examples:}		
		\begin{itemize}
			\item  lag\_days([5, 3, 2, 1]) $\rightarrow$ 3, 
				(3 lag days, day2 since (3$<$5), day3 since (2$<$3), and day4 since (1$<$2))
			\item  lag\_days([10, 11, 12, 9, 10]) $\rightarrow$ 1, 
			\item  lag\_days([6, 5, 4, 3, 2, 9]) $\rightarrow$ 4, 
			\item  lag\_days([9, 9]) $\rightarrow$ 0
		\end{itemize}

\end{enumerate}
\pagebreak
\end{document}