%standard 9.2
%week 8


%start_of_questions



%new_question
%%%%%%%%%%%%%%%%%%%%%
	% Problem 12
	% Difficulty: 2
%%%%%%%%%%%%%%%%%%%%%
	\item 
		Write a \textbf{function} named \textit{is\_even} that returns a boolean value which determines 
		if an integer is even.  Write a second function named \textit{report\_evens} that takes a list 
		of integers and returns a new list containing all the even numbers from the original list. 
		Call the \textit{is\_even} function as part of the \textit{report\_evens} function. 
		
		\begin{itemize}
			\item  report\_evens([4,3,12,16,8,9,25]) $\rightarrow$ [4,12,16,8]
			\item  report\_evens([6,100,3,12,16,6,9,100]) $\rightarrow$ [6,100,12,16,6,100]
			\item  report\_evens([3,99,7,13,25]) $\rightarrow$ []
		\end{itemize}


%new_question
%%%%%%%%%%%%%%%%%%%%%
	% Problem 13
	% Difficulty: 2
%%%%%%%%%%%%%%%%%%%%%
	\item 
		Write a \textbf{function} named \textit{is\_vowel} that returns a boolean value which determines 
		if an letter is a vowel.  Write a second function named \textit{report\_vowels} that takes a string 
		and returns a list containing all the vowels from the original string.
		Call the \textit{is\_vowel} function as part of the \textit{report\_vowels} function. 

		Hint: In the English language, the letters a, e, i, o, and u are the vowels.\\
		\textbf{Examples:}		
		\begin{itemize}
			\item report\_vowels(\csq{apple}) $\rightarrow$ [a,e]
			\item report\_vowels(\csq{banana}) $\rightarrow$ [a,a,a] 
			\item report\_vowels(\csq{run time error}) $\rightarrow$ [r,i,e,e,o]
		\end{itemize}

	

%new_question
%%%%%%%%%%%%%%%%%%%%%
	% Problem 14
	% Difficulty: 2
%%%%%%%%%%%%%%%%%%%%%
	\item 
		Write a \textbf{function} named \textit{is\_two\_digit\_number} that returns a boolean value 
		which determines if an integer is a two digit number. Write a second function named 
		\textit{report\_two\_digit\_numbers} that takes a list of integers and returns a new list 
		containing all the two digit numbers from the original list. 
		Call the \textit{is\_two\_digit\_number} function as part of the \textit{report\_two\_digit\_numbers} function. 

		Hint: a two digit number is one in the range $[-99,-10]\cup[10,99]$.\\
		\textbf{Examples:}		
		\begin{itemize}
			\item report\_two\_digit\_numbers([100,57,12,1]) $\rightarrow$ [57,12]
			\item report\_two\_digit\_numbers([121,36,-19,-6,0,21]) $\rightarrow$ [36,-19,21]
			\item report\_two\_digit\_numbers([100,7,8437]) $\rightarrow$ []
		\end{itemize}



%new_question
%%%%%%%%%%%%%%%%%%%%%
	% Problem 15
	% Difficulty: 2
%%%%%%%%%%%%%%%%%%%%%
	\item 
		Write a function named \textit{is\_negative} that returns a boolean value which determines 
		if an integer is a negative number. 
		Write a second function named \textit{is\_odd} that returns a boolean value which determines 
		if an integer is odd.
		Write a third function named \textit{report\_negative\_odds} that takes a list of integers 
		and returns a new list containing all the negative odd numbers from the original list. 
		The \textit{report\_negative\_odds} function must call the \textit{is\_negative} and 
		\textit{is\_odd} to	determine if an element belongs.
		
		\textbf{Examples:}		
		\begin{itemize}
			\item report\_negative\_odds([100,-57,12,1,-36,-15]) $\rightarrow$ [-57,-15]
			\item report\_negative\_odds([121,-101,36,-19,-6,0,21,-1]) $\rightarrow$ [-101,-19,-1]
			\item report\_negative\_odds([-100,7,8437]) $\rightarrow$ []
		\end{itemize}




%end_of_questions
%make sure to leave at least one blank line below

