%standard 13.3



%start_of_questions


%new_question
%%%%%%%%%%%%%%%%%%%%%
	% Problem 8
	% Difficulty: 3
%%%%%%%%%%%%%%%%%%%%%
	\item % give playlist default name.
		Write a class for a \textbf{Playlist} with the instance variables and methods listed 
		below.\\
		A Playlist should have a default name of \csq{New Playlist}.\\
		It can be instantiated with initial songs, but it is not required to.\\
		Create a method called \textit{add\_song} which adds a song title (a string) to the 
		Playlist.\\
		You should be able to combine two Playlists, and print them in a readable way.
			
		\begin{minipage}[t]{0.65\textwidth}
			For example:
			\begin{itemize}
				\item $p_1 = [\csqt{Song A}, \csqt{Song B}]$
				\item $p_2 = [\csqt{Song C}]$
				\item $p_1 + p_2 = [\csqt{Song A}, \csqt{Song B}, \csqt{Song C}]$
			\end{itemize}

			Your class should support:
			\begin{itemize}
				\item Creating a playlist with a name and list of songs
				\item Adding two playlists (combines song lists)
				\item Printing the playlist in a readable way (e.g., list songs)
			\end{itemize}	
		\end{minipage}
		\hfill
		\begin{minipage}[t]{0.32\textwidth}
			\vspace{.2em}
			\begin{flushright}
				\begin{tabular}{|l|}
					\hline
					Playlist \\ \hline
					name \\
					songs (list of strings) \\ \hline
					\_\_init\_\_ \\
					add\_song \\
					\_\_add\_\_ \\
					\_\_str\_\_ \\ \hline
				\end{tabular}
			\end{flushright}
		\end{minipage}
		
		Once you have created the class, add code that:
		\begin{itemize}
			\item Creates two playlists and at least one song to each.
			\item Combines the playlists
			\item Prints the result
		\end{itemize}



%new_question
%%%%%%%%%%%%%%%%%%%%%
	% Problem 9
	% Difficulty: 3
%%%%%%%%%%%%%%%%%%%%%
	\item
		Write a class for a \textbf{ShoppingCart} with the instance variables and methods listed 
		below.\\
		It can be instantiated with initial items in the cart, but it is not required to.\\
		Create a method called \textit{add\_items} which adds an item (a string) to the 
		ShoppingCart. The same item can be added more than once. If the item is already in the 
		ShoppingCart, increase its quantity by one. If its not in the ShoppingCart, set the 
		quantity to 1.\\ Hint: think about what type of data structure you should use. \\
		When two ShoppingCarts are added together, the result should be a ShoppingCart 
		containing the items in both. Overlapping items should have a sum of their quantities.

		\begin{minipage}[t]{0.65\textwidth}
			For example:
			\begin{itemize}
				\item $p_1 = \{\csqt{tea}:1, \csqt{energy drink}:2 \}$
				\item $p_2 = \{\csqt{energy drink}:3, \csqt{hat}:1 \}$
				\item $p_1 + p_2 = \{\csqt{tea}:1, \csqt{energy drink}:5, \csqt{hat}:1 \}$
			\end{itemize}		
		\end{minipage}
		\hfill
		\begin{minipage}[t]{0.32\textwidth}
			\vspace{0.1em}
			\begin{flushright}
				\begin{tabular}{|l|} \hline
					ShoppingCart \\ \hline
					items (dict$\rightarrow$str:int) \\ \hline
					\_\_init\_\_ \\
					add\_item \\
					\_\_add\_\_ \\
					\_\_str\_\_ \\ \hline
				\end{tabular}
			\end{flushright}
		\end{minipage}
		
		Your class should support:
		\begin{itemize}
			\item Creating a ShoppingCart
			\item Adding two ShoppingCarts (combines items)
			\item Printing the ShoppingCart in a readable way 
				(e.g., lists all items with quantities)
		\end{itemize}
		
		Once you have created the class, add code that:
		\begin{itemize}
			\item Creates two ShoppingCarts and at least one item to each.
			\item Combines the ShoppingCarts
			\item Prints the result
		\end{itemize}

%end_of_questions
%make sure to leave at least one blank line below

