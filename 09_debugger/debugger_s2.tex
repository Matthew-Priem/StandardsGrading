%standard 10.2


%start_of_questions


%new_question
%%%%%%%%%%%%%%%%%%%%%
	% Problem 8
	% Difficulty: 1
%%%%%%%%%%%%%%%%%%%%%
	\item 
		Given a positive integer $n$, the following rules will always create a sequence that 
		ends with 1, called Hailstone Sequence:
		\begin{enumerate}
			\item If $n$ is even, divide by 2
			\item If $n$ is odd, multiply by 3 and add 1 (i.e. $3n+1$)
			\item Continue until $n$ is 1
		\end{enumerate}
		Write a \textbf{function} that returns a list with the hailstone sequence starting at $n$. 
		The argument to the function will be $n$ (the integer to start the sequence from).\\

		\textbf{Debug this Solution:}\\
		\mbox{ \hspace*{0.25in}	\lstinputlisting[language=Python]{./code/debugger_p9_HailstoneSequence.py}}

%Error 1: n/n should be n
%Error 2: sequence.append(n) should have an extra indentation
%Error 3: while n == 1 should be while != 1
\pagebreak



%new_question
%%%%%%%%%%%%%%%%%%%%%
	% Problem 9
	% Difficulty: 1
%%%%%%%%%%%%%%%%%%%%%
	\item
		YouTube currently displays a like and a dislike button, allowing you to express your opinions 
		about particular content. 
		It's set up in such a way that you cannot like and dislike a video at the same time.
		There are two other interesting rules to be noted about the interface:
		\begin{enumerate}
			\item Pressing a button, which is already active, will undo your press.
			\item If you press the like button after pressing the dislike button, the like button overwrites 
				the previous \csq{dislike} state. The same is true for the other way round.
		\end{enumerate}
		Write a \textbf{function} that takes in a list of button inputs $events$ and returns the final state.\\

		\textbf{Debug this Solution:}\\
		\mbox{ \hspace*{0.25in}	\lstinputlisting[language=Python]{./code/debugger_p10_YouTube.py}}

%Error 1: initial state = "like" should be state = "nothing"
%Error 2: for event in range(events) should be for event in events
%Error 3: if event != state: should be if event == state:
\pagebreak



%new_question
%%%%%%%%%%%%%%%%%%%%%
	% Problem 10
	% Difficulty: 1
%%%%%%%%%%%%%%%%%%%%%
	\item 	
		%https://edabit.com/challenge/yL5WmWTCNwwb4GnR7
		In each input list, every number repeats at least once, except for two. Write a \textbf{function} 
		that takes an array $numbers$ and returns the two unique numbers.\\

		\textbf{Debug this Solution:}\\
		\mbox{ \hspace*{0.25in}	\lstinputlisting[language=Python]{./code/debugger_p11_TwoUniqueNumbers.py}}

%Error 1: for num in range(len(numbers)) should be for num in numbers
%Error 2: code in if and else statements should be swapped.
%Error 3: number_dicitonary.values() should be number_dicitonary
\pagebreak



%new_question
%%%%%%%%%%%%%%%%%%%%%
	% Problem 11
	% Difficulty: 1
%%%%%%%%%%%%%%%%%%%%%
	\item 
		%https://edabit.com/challenge/6Pf5GGG6HnzbB95gf
		Write a \textbf{function} that returns a list with the factors of a given integer. 
		The argument of the function will be $num$ (integer to find factors for).\\

		\textbf{Debug this Solution:}\\
		\mbox{ \hspace*{0.25in}	\lstinputlisting[language=Python]{./code/debugger_p13_Factors.py}}

%Error 1: for i in range(1, num): should be for i in range(1, num+1):
%Error 2:  if num % i != 0: should be if num % i == 0:
%Error 3: factors.add(i) should be factors.append(i)
\pagebreak




%new_question
%%%%%%%%%%%%%%%%%%%%%
	% Problem 12
	% Difficulty: 1
%%%%%%%%%%%%%%%%%%%%%
\item
	Write a \textbf{function} that takes a list of words $words$ and returns a dictionary where 
	the keys categorize words based on whether they are palindromes. 
	The categories are defined as follows:
	\begin{enumerate}  
		\item \csq{Palindrome} includes words that read the same forward and backward.  
		\item \csq{Non-palindrome} includes all other words.  
	\end{enumerate}  

		\textbf{Debug this Solution:}\\
		\mbox{ \hspace*{0.25in}	\lstinputlisting[language=Python]{./code/debugger_p14_Palindromes.py}}

%Error 1: reduce indent of if and else
%Error 2: put "reversed_word = ' ' '' inside first loop
%Error 3: switch content of if and else OR change if statement from == to !=
\pagebreak



%new_question

%%%%%%%%%%%%%%%%%%%%%
	% Problem 13
	% Difficulty: 1
%%%%%%%%%%%%%%%%%%%%%
	\item
		(Game: Odd or Even)  Write a \textbf{function} that lets the user guess whether a randomly 
		generated number is odd or even.  The function randomly generates an integer between 0 and 9 
		(inclusive) and returns whether the user's guess is correct or incorrect. The argument for 
		the function will be $guess$ (the user's guess, either \csq{odd} or \csq{even}), if no 
		argument is provided then the \textbf{default} guess should be even.\\
		Hint: Use the following lines of code to create the function.
		\begin{verbatim}
		    from random import randint
		    value = randint(0,9) #picks a random integer between 0-9 inclusive
		\end{verbatim}

		\textbf{Debug this Solution:}\\
		\mbox{ \hspace*{0.25in}	\lstinputlisting[language=Python]{./code/debugger_p15_EvenOrOdd.py}}


%Error 1: from random import randominteger should be from random import randomint
%Error 2: def guess(guess="odd"): should be def guess(guess="even"):
%Error 3: if value // 2 == 0: should be if value % 2 == 0:
\pagebreak


%end_of_questions
%make sure to leave at least one blank line below

