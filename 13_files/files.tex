\documentclass{article}

\usepackage{amsmath}
\usepackage{amsfonts} % For math fonts.
\usepackage{amssymb} % For other math symbols not covered by amsmath.
\usepackage[pdftex]{graphicx} % For pictures, use %\includegraphics[scale=decimal]{pic.png}; must be a .png file type.
\usepackage{multicol}
\usepackage{textcomp}
\usepackage[colorlinks = true, urlcolor = blue]{hyperref}
\usepackage{enumitem}
\usepackage{graphbox} 
\usepackage{subfig}
\usepackage{multicol}
\usepackage{wrapfig}


\newcommand{\tab}{\hspace*{0.25in}}

\usepackage{tikz}
\usetikzlibrary{positioning, calc}
\usetikzlibrary{shapes.geometric,angles,quotes}
\usepackage{tikz-3dplot}

\newcommand{\csq}[1]{\reflectbox{''}#1''}  %This produces CS style quotes.



\usepackage{fullpage}
\usepackage{listings}
\lstset
{ %Formatting for code in appendix
    language=Python,
    basicstyle=\footnotesize,
    numbers=left,
    stepnumber=1,
    showstringspaces=false,
    tabsize=2,
    breaklines=true,
    breakatwhitespace=false,
}


\begin{document}


\begin{flushright}
Intro functions\end{flushright}

\vspace*{-1.5em}
\noindent\makebox[\linewidth]{\rule{\paperwidth}{0.4pt}}


\vspace*{2em}

\begin{enumerate}

%standard 14.1

%start_of_questions


%new_question
%%%%%%%%%%%%%%%%%%%%%
	% Problem 1
	% Difficulty: 1
%%%%%%%%%%%%%%%%%%%%%
	\item 
		Write 100 integers created randomly into a file named \textit{QuizInts.txt}. 
		The numbers should be between 50 and 200 (inclusively). 
		Each number should be on a new line.\\
		Hint: Your code will likely use the following two lines of code somewhere in your program.\\
			\tab import random.\\
			\tab random.randint(50,200)

%new_question
%%%%%%%%%%%%%%%%%%%%%
	% Problem 2
	% Difficulty: 1
%%%%%%%%%%%%%%%%%%%%%			
	\item 
		Write a Python program that will open a file named \textit{thisFile.txt} and write every 
		other line into the file
		\textit{thatFile.txt}

%new_question
%%%%%%%%%%%%%%%%%%%%%
	% Problem 3
	% Difficulty: 1
%%%%%%%%%%%%%%%%%%%%%
	\item
		Create a file named \textit{MyName.txt}, and write your name to it (your actual name).	 
		Then read the file and print the letters of your name one at a time where each letter is on a new line.
		\begin{figure}[ht]
			\centering
			\begin{minipage}[b]{.4\textwidth}
				\centering
				\includegraphics[scale=1]{imgs/nameFile.png}
				\caption{This is the file.}	
			\end{minipage}
			\hspace*{2em}
			\begin{minipage}[b]{.4\textwidth}
				\centering
				\includegraphics[width=1\textwidth]{imgs/nameOutput.png}
				\caption{This is the output.}
			\end{minipage}
		\end{figure}


%new_question
%%%%%%%%%%%%%%%%%%%%%
	% Problem 4
	% Difficulty: 1
%%%%%%%%%%%%%%%%%%%%%		
	\item
		Assume you are working on a file named \textit{MyCode.py} and there is a file \textit{MyWords.txt} in the same 
		working directory (same folder). The \textit{MyWords.txt} file contains exactly 20 words all written on separate
		lines. Read the file, and then write the words to a new file in four lines of five words.



%new_question
%%%%%%%%%%%%%%%%%%%%%
	% Problem 6
	% Difficulty: 1
%%%%%%%%%%%%%%%%%%%%%
	\item
		Assume you have a text file called \textit{aMorePerfectUnion.txt} that contains a transcript 
		of Barack Obama's March $18^{th}$, 2008 speech \textit{A More Perfect Union}. Create a 
		dictionary consisting each word and the amount of times that word appears in the speech. 
		Print the dictionary.

%end_of_questions
%make sure to leave at least one blank line below


%standard 14.2

%start_of_questions




%new_question
%%%%%%%%%%%%%%%%%%%%%
	% Problem 5
	% Difficulty: 2
%%%%%%%%%%%%%%%%%%%%%
	\item 
		A local middle school is trying to count the total number of lunches they served last year.  
		They have a text file named \textit{LunchData.txt} that has a date and the number of lunches served on that date.    
		There is one entry for every day last year.  A portion of that file is displayed below.  
		Write a program that calculates and then prints the total number of lunches served last year. 
		\begin{flushright}
			\includegraphics[scale=.65]{imgs/LunchData.PNG}
		\end{flushright}




%new_question
%%%%%%%%%%%%%%%%%%%%%
	% Problem 7
	% Difficulty: 2
%%%%%%%%%%%%%%%%%%%%%
	\item 
		A city library keeps track of the number of visitors each day in a file named 
		\textit{LibraryVisitsData.csv}.  
		The file contains a date and the number of visitors who entered the library on that date.  
		There is one entry for each day of the year. A portion of that file is shown below.  
		Write a program that reads the file, calculates, and prints the average number of visitors 
		per day over the year.
		
		\begin{flushright}
			\includegraphics[scale=.65]{imgs/LibraryVisitsData.PNG}
		\end{flushright}


%new_question
%%%%%%%%%%%%%%%%%%%%%
	% Problem 8
	% Difficulty: 2
%%%%%%%%%%%%%%%%%%%%%
	\item 
		A local gym keeps a log of how many calories were burned in workout sessions each day, stored in a file called \textit{CaloriesBurnedData.txt}.  
		Each line of the file includes the date and the total number of calories burned by all gym members on that day.  
		A portion of the file is shown below.  
		Write a program that reads the file and prints the day with the highest number of calories burned.
		
		\begin{flushright}
			\includegraphics[scale=.65]{imgs/CaloriesBurnedData.PNG}
		\end{flushright}


%new_question
%%%%%%%%%%%%%%%%%%%%%
	% Problem 9
	% Difficulty: 2
%%%%%%%%%%%%%%%%%%%%%
	\item 
		A school science fair recorded the daily number of students who visited each exhibit. 
		This information is stored in a file called \textit{ScienceFairVisitors.txt}, which includes 
		a header row. Each line contains the date and the number of visitors for that day. 
		Write a program that reads the file, and then prints the total number of visitors recorded 
		over the entire period.
		
		\begin{flushright}
			\includegraphics[scale=.65]{imgs/ScienceFairVisitorsData.PNG}
		\end{flushright}


%new_question
%%%%%%%%%%%%%%%%%%%%%
	% Problem 10
	% Difficulty: 2
%%%%%%%%%%%%%%%%%%%%%
	\item 
		A book club tracks how many pages each member read, stored in a file named 
		\textit{PagesRead.csv}. The file includes a header row and contains the member's 
		name and the number of pages they read for each book.  
		Write a program that reads the file, stores the data in a dictionary where the key is 
		the member name and the value is the total pages read by that member (across both books), 
		and then prints each member's name and their total pages read.
		
		\begin{flushright}
			\includegraphics[scale=.65]{imgs/PagesReadData.PNG}
		\end{flushright}


%new_question
%%%%%%%%%%%%%%%%%%%%%
	% Problem 11
	% Difficulty: 2
%%%%%%%%%%%%%%%%%%%%%
	\item 
		A music streaming app tracks how many times each user listens to different songs.  
		The data is stored in a file called \textit{SongPlays.txt}, which includes a header row.  
		Each line contains the user’s name and the number of times they played a song on a given day.  
		Write a program that reads the file, uses a dictionary to store the total plays per user, 
		and then prints out each user and their total number of song plays.
		
		\begin{flushright}
			\includegraphics[scale=.65]{imgs/SongPlaysData.PNG}
		\end{flushright}


%new_question
%%%%%%%%%%%%%%%%%%%%%
	% Problem 12
	% Difficulty: 2
%%%%%%%%%%%%%%%%%%%%%
	\item 
		A weather station logs the temperature each day and stores the data in a file called 
		\textit{DailyTemperatures.csv}. The file includes a header row and each line contains the date 
		and the temperature recorded on that day. Write a program that reads the file, stores all the 
		temperatures in a list, and then prints the highest, lowest, and average temperature recorded.
		
		\begin{flushright}
			\includegraphics[scale=.65]{imgs/DailyTemperaturesData.PNG}
		\end{flushright}


%end_of_questions
%make sure to leave at least one blank line below


%standard 14.3

%start_of_questions



%new_question
%%%%%%%%%%%%%%%%%%%%%
	% Problem 13
	% Difficulty: 3
%%%%%%%%%%%%%%%%%%%%%
	\item 
		Create a python program that writes the name and age of everyone in your family to .csv file.
		There should be a column for the name with a header titled Name, and there should be a column 
		for the age with a header titled Age.
		Do not use the csv module. You may make up fake family members if you choose. The result
		should look similar to the following.
		
		\begin{flushright}
			\includegraphics[scale=.65]{imgs/WritingFamily.PNG}
		\end{flushright}



%Write a few that write to write with multiple columns using a clear delimeter. 



%end_of_questions
%make sure to leave at least one blank line below



\end{enumerate}
\end{document}



%new_question
%%%%%%%%%%%%%%%%%%%%%
	% Problem 5
%%%%%%%%%%%%%%%%%%%%%
	\item
		A local business is trying to count the total number of customer they had last year.  
		They have a text file named \textit{CustomerData.txt} that has a date and the number of customers on that date.    
		There is one entry for every day this year.  A potion of that file is displayed below.  
		Write a program that calculates and then prints the total number of customers the business had last year.\\ 
		\begin{flushright}
			\fbox{\includegraphics[scale=1.5]{imgs/ScienceFairVisitorsData.PNG}}
		\end{flushright}

%new_question
%%%%%%%%%%%%%%%%%%%%%
	% Problem 5
%%%%%%%%%%%%%%%%%%%%%
	%paper-based
	\item
		Write a program that will open a file name "thisFile.txt" and write every other line into the file "thatFile.txt".\\ 

%new_question
%%%%%%%%%%%%%%%%%%%%%
	% Problem 5
%%%%%%%%%%%%%%%%%%%%%
	%paper-based
	\item
		Write a program that reads and prints each line of a file named "TheFile.txt".\\ 

%new_question
%%%%%%%%%%%%%%%%%%%%%
	% Problem 5
%%%%%%%%%%%%%%%%%%%%%
	%paper-based
	\item
		Write a program that writes each letter of your first name to a file named "NameFile.txt". Each letter should be on a new line.
		Use your actual first name.\\ 

%new_question
%%%%%%%%%%%%%%%%%%%%%
	% Problem 5
%%%%%%%%%%%%%%%%%%%%%
	%paper-based
	\item
		This question has two parts. You will create a file in part (a), and then use the created file in part (b). The two parts
		are meant to be independent. If you don't know how to create the file in part (a), you may assume its been created
		correctly for part (b).\\
		\begin{enumerate}
			\item
				Create a file named $numbers.txt$ and write to it 100 randomly created integers between 250 and 500
				(inclusively). Each number should be written on a new line.\\
				Hint: Don't forget to import random.
			
			\item
				Cycle through each number in your newly created file, and print the word $even$ if it is even. Print the word $odd$
				if it is odd.

		 



