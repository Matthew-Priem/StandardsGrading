%standard 15.2 

%start_of_questions




%%%%%%%%%%%%%%%%%%%%%
	% Problem 5
	% Difficulty: 2
%%%%%%%%%%%%%%%%%%%%%
%new_question
	\item 
		You are helping a teacher update students' scores after a quiz.  
		The teacher wants to add points for extra credit and needs your program to 
		do the math safely.\\

		A dictionary stores the number of points each student has earned, shown below: 
		\begin{itemize}
			\item \texttt{\{\csq{Alice}: 90, \csq{Bob}: 75, \csq{Charlie}: 60\}}
		\end{itemize}
		Write a \textbf{program} that asks the user to enter a student's name and a number 
		to add to their score. The program should print the new number of points.

		The program should handle the following errors:
		\begin{itemize}
			\item If the name is not found in the dictionary (\textbf{KeyError}), 
				print \csq{Student not found.}
			\item If the number entered is not valid (e.g., not a number) (\textbf{ValueError}), 
				print \csq{Invalid number.}
		\end{itemize}

		\textbf{Examples:}
		\begin{itemize}
			\item \csq{Enter student name:} \csq{Bob} \\
			      \csq{Enter number to add:} 10 $\rightarrow$ 85
			\item \csq{Enter student name:} \csq{David} $\rightarrow$ \csq{Student not found.}
			\item \csq{Enter student name:} \csq{Alice} \\
			      \csq{Enter number to add:} \csq{ten} $\rightarrow$ \csq{Invalid number.}
		\end{itemize}

%Errors: KeyError, ValueError




%%%%%%%%%%%%%%%%%%%%%
	% Problem 6
	% Difficulty: 2
%%%%%%%%%%%%%%%%%%%%%
%new_question
	\item 
		You're building a simple scheduling tool that lets users select a day of the week by 
		entering a number between 0 and 6.  
		Each number corresponds to a day, starting with 0 for Monday and ending with 6 for Sunday.\\

		A list contains the days of the week, shown below: 
		\begin{itemize}
			\item \texttt{[\csq{Monday}, \csq{Tuesday}, \csq{Wednesday}, \csq{Thursday}, 
				\csq{Friday}, \csq{Saturday}, \csq{Sunday}]}.  
		\end{itemize}
		Write a \textbf{program} that asks the user to enter a number (0--6) and prints the 
		corresponding day.

		The program should handle the following errors:
		\begin{itemize}
			\item If the input is not a valid number (\textbf{ValueError}), 
				print \csq{Invalid input.}
			\item If the number is outside the valid range (\textbf{IndexError}), 
				print \csq{Index out of range.}
		\end{itemize}

		\textbf{Examples:}
		\begin{itemize}
			\item \csq{Enter a number:} 0 $\rightarrow$ \csq{Monday}
			\item \csq{Enter a number:} 6 $\rightarrow$ \csq{Sunday}
			\item \csq{Enter a number:} 7 $\rightarrow$ \csq{Index out of range.}
			\item \csq{Enter a number:} \csq{two} $\rightarrow$ \csq{Invalid input.}
		\end{itemize}

%Errors: IndexError, ValueError


%%%%%%%%%%%%%%%%%%%%%
	% Problem 7
	% Difficulty: 2
%%%%%%%%%%%%%%%%%%%%%
%new_question
	\item 
		You're building a tool that compares two numbers by calculating both the difference 
		and the ratio. The program should ask the user to enter two numbers and then:
		\begin{itemize}
			\item Print the difference (first minus second)
			\item Print the result of dividing the first number by the second
		\end{itemize}

		Write a \textbf{program} that uses \texttt{int()} to convert user input and 
		performs both calculations.  

		The program should handle the following errors:
		\begin{itemize}
			\item If either input is not a valid number (\textbf{ValueError}), 
				print \csq{Invalid input.}
			\item If the second number is 0 (\textbf{ZeroDivisionError}), 
				print \csq{Cannot divide by zero.}
			\item If the result is too large (\textbf{OverflowError}), 
				print \csq{Result too large.}
		\end{itemize}

		\textbf{Examples:}
		\begin{itemize}
			\item \csq{Enter first number:} 100 \\
			      \csq{Enter second number:} 20 $\rightarrow$ \csq{Difference: 80, Ratio: 5.0}
			\item \csq{Enter first number:} 42 \\
			      \csq{Enter second number:} 0 $\rightarrow$ \csq{Cannot divide by zero.}
			\item \csq{Enter first number:} \csq{ten} $\rightarrow$ \csq{Invalid input.}
			\item \csq{Enter first number:} 999999999999999999999999 \\
			      \csq{Enter second number:} 1 $\rightarrow$ \csq{Result too large.}
		\end{itemize}

%Errors: ValueError, ZeroDivisionError, OverflowError


%end_of_questions
%make sure to leave at least one blank line below

