%standard 7.3
%week6

%start_of_questions



%new_question
%%%%%%%%%%%%%%%%%%%%%
	% Problem 9
	% Difficulty: 3
%%%%%%%%%%%%%%%%%%%%%
%https://edabit.com/challenge/hYiCzkLBBQSeF8fKF
	\item 
		In BlackJack, cards are counted with -1, 0, 1 values:
		\begin{itemize}
			\item 2, 3, 4, 5, 6 are counted as +1
			\item 7, 8, 9 are counted as 0
			\item 10, j, q, k, a are counted as -1
		\end{itemize}
		Write a \textbf{function} that takes a list called $cards$, counts the number, 
		and returns it from the list of cards provided.

		\textbf{Examples:}		
		\begin{itemize}
			\item  count([5, 9, 10, 3, \csq{j}, \csq{a}, 4, 8, 5]) $\rightarrow$ 1, 
			\item  count([\csq{a}, \csq{a}, \csq{k}, \csq{q}, \csq{q}, \csq{j}]) $\rightarrow$ -6, 
			\item  count([\csq{a}, 5, 5, 2, 6, 2, 3, 8, 9, 7]) $\rightarrow$ 5
		\end{itemize}



%new_question
%%%%%%%%%%%%%%%%%%%%%
	% Problem 10
	% Difficulty: 3
%%%%%%%%%%%%%%%%%%%%%
%https://edabit.com/challenge/rMr8yRxS8TeF9pDyn
	\item 
		There's a great war between the even and odd numbers. Many numbers already lost 
		their lives in this war and it's your task to end this. You have to determine 
		which group sums larger: the evens or the odds. The larger group wins.

		Write a \textbf{function} that takes a list of integers named \textit{numbers}, 
		sums the even numbers and odd numbers separately, then returns which of the two
		sums is larger.

		\textbf{Examples:}		
		\begin{itemize}
			\item  war\_of\_numbers([2, 8, 7, 5]) $\rightarrow$ 
				\csq{odds}, (since 2 + 8 = 10, 7 + 5 = 12, odds is larger) 
			\item  war\_of\_numbers([12, 90, 75, 1, 1]) $\rightarrow$ \csq{evens}, 
				(12 + 90 = 102, 75 + 1 + 1 = 77, evens is larger) 
			\item  war\_of\_numbers([2, 10, 22, 243]) $\rightarrow$ \csq{odds}
		\end{itemize}



%new_question
%%%%%%%%%%%%%%%%%%%%%
	% Problem 14
	% Difficulty: 3
%%%%%%%%%%%%%%%%%%%%%
%https://edabit.com/challenge/2yHQwkecEHZBfHcmN
	\item
		To train for an upcoming marathon, Johnny goes on one long-distance run each Saturday. 
		He wants to track how often the number of miles he runs exceeds the previous Saturday. 
		This is called a progress day. Write a \textbf{function} that takes in a list of miles 
		run every Saturday and returns Johnny's total number of progress days.

		\textbf{Examples:}		
		\begin{itemize}
			\item  progress\_days([3, 4, 1, 2]) $\rightarrow$ 2, 
				(Two progress days, day 2 since $(4>3)$ and day 4 since $(2>1)$)
			\item  progress\_days([10, 11, 12, 9, 10]) $\rightarrow$ 3, 
			\item  progress\_days([6, 5, 4, 3, 2, 9]) $\rightarrow$ 1, 
			\item  progress\_days([9, 9]) $\rightarrow$ 0
		\end{itemize}

%new_question
%%%%%%%%%%%%%%%%%%%%%
	% Problem 15
	% Difficulty: 3
%%%%%%%%%%%%%%%%%%%%%
	\item
		To train for an upcoming marathon, Samuel goes on one long-distance run each Saturday. 
		He wants to track how often the number of miles he runs fall short of the previous Saturday. 
		This is called a lag day. Write a \textbf{function} that takes in a list of miles 
		run every Saturday and returns Samuel's total number of lag days.

		\textbf{Examples:}		
		\begin{itemize}
			\item  lag\_days([5, 3, 2, 1]) $\rightarrow$ 3, 
				(3 lag days, day2 since (3$<$5), day3 since (2$<$3), and day4 since (1$<$2))
			\item  lag\_days([10, 11, 12, 9, 10]) $\rightarrow$ 1, 
			\item  lag\_days([6, 5, 4, 3, 2, 9]) $\rightarrow$ 4, 
			\item  lag\_days([9, 9]) $\rightarrow$ 0
		\end{itemize}

%new_question
%%%%%%%%%%%%%%%%%%%%%
	% Problem 19
	% Difficulty: 3
%%%%%%%%%%%%%%%%%%%%%%https://leetcode.com/problems/check-if-a-string-is-an-acronym-of-words/description/
	\item 
		Let $\bm{s}$ be a string and $\bm{words}$ be a list of strings. The string $\bm{s}$ is 
		considered an acronym of $\bm{words}$ if it can be formed by concatenating the first 
		character of each string in $\bm{words}$ in order. For example, \texttt{"ab"} can be formed 
		from [\texttt{"apple"}, \texttt{"banana"}], but it can't be formed from [\texttt{"bear"}, 
		\texttt{"aardvark"}]. Write a function that takes a string $\bm{s}$ and a list of strings $
		\bm{words}$, and returns True if $\bm{s}$ is an acronym of $\bm{words}$, and False 
		otherwise.

		\textbf{Examples:}		
		\begin{itemize}
			\item  is\_acronym(\texttt{"abc"}, [\texttt{"alice"}, \texttt{"bob"}, \texttt{"charlie"}] ) 
				$\rightarrow$ True\\
				Explanation: The first character in the words \texttt{"alice"}, \texttt{"bob"}, and 
				\texttt{"charlie"} are \texttt{"a"}, \texttt{"b"}, and \texttt{"c"}, respectively. 
				Hence, s = \texttt{"abc"} is the acronym. 			
			\item  is\_acronym(\texttt{"a"}, [\texttt{"an"}, \texttt{"apple"}] ) $\rightarrow$ False\\
				Explanation: The first character in the words \texttt{"an"} and \texttt{"apple"} are 
				\texttt{"a"} and \texttt{"a"}, respectively. The acronym formed by concatenating these 
				characters is \texttt{"aa"}. Hence, s = \texttt{"a"} is not the acronym.
			\item  is\_acronym(\texttt{"ngguoy"}, [\texttt{"never"}, \texttt{"gonna"}, \texttt{"give"}, 
				\texttt{"up"}, \texttt{"on"}, \texttt{"you"}]) $\rightarrow$ True\\\
				Explanation: By concatenating the first character of the words in the array, we get the 
				string \texttt{"ngguoy"}. Hence, s = \texttt{"ngguoy"} is the acronym.
			\item  is\_acronym(\texttt{"ab"}, [\texttt{"apple"}, \texttt{"banana"}, \texttt{cat}]) 
				$\rightarrow$ False\\
				Explanation: Wrong length
			\item  is\_acronym(\texttt{"ab"}, [\texttt{"apple"}, \texttt{""}, \texttt{cat}]) 
				$\rightarrow$ False\\
				Explanation: Can't get the first letter in the second string of $\bm{words}$.
	
		\end{itemize}

	%Explanation: By concatenating the first character of the words in the array, we get the string "ngguoy". 
	%Hence, s = "ngguoy" is the acronym.


%end_of_questions
%make sure to leave at least one blank line below

