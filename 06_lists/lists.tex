\documentclass{article}

\usepackage{amsmath}
\usepackage{amsfonts} % For math fonts.
\usepackage{amssymb} % For other math symbols not covered by amsmath.
\usepackage[pdftex]{graphicx} % For pictures, use \includegraphics[scale=decimal]{pic.png}; must be a .png file type.
\usepackage{multicol}
\usepackage{textcomp}
\usepackage[colorlinks = true, urlcolor = blue]{hyperref}
\usepackage{enumitem}
\usepackage{graphbox} 
\usepackage{subfig}
\usepackage{multicol}
\usepackage{bm}

\newcommand{\tab}{\hspace*{0.25in}}
\newcommand{\csq}[1]{\reflectbox{''}#1''}  %This produces CS style quotes.

\usepackage{tikz}
\usetikzlibrary{positioning, calc}
\usetikzlibrary{shapes.geometric,angles,quotes}


\usepackage{fullpage}
\usepackage{listings}
\lstset
{ %Formatting for code in appendix
    language=Python,
    basicstyle=\footnotesize,
    numbers=left,
    stepnumber=1,
    showstringspaces=false,
    tabsize=2,
    breaklines=true,
    breakatwhitespace=false,
}


\begin{document}


\begin{flushright}
loops\end{flushright}

\vspace*{-1.5em}
\noindent\makebox[\linewidth]{\rule{\paperwidth}{0.4pt}}


\vspace*{2em}

\begin{enumerate}


%standard 7.1
%week 6



%start_of_questions

%new_question
%%%%%%%%%%%%%%%%%%%%%
	% Problem 1
	% Difficulty: 1
%%%%%%%%%%%%%%%%%%%%%
	\item 
		Write a \textbf{function} that loops through a word and returns a list with every 
		other letter of the word starting with the \textbf{first} letter.
		The function will take a single argument $word$ (a string representing the word to process).

		\textbf{Examples:}		
		\begin{itemize}
			\item  skip\_letter(\csq{counterattack}) $\rightarrow$ 
				[\csq{c},\csq{u},\csq{t},\csq{r},\csq{t},\csq{a},\csq{c}]
			\item  skip\_letter(\csq{banana sunday}) $\rightarrow$
				[\csq{b},\csq{n},\csq{n},\csq{s},\csq{n},\csq{a}]
		\end{itemize}

%new_question
%%%%%%%%%%%%%%%%%%%%%
	% Problem 2
	% Difficulty: 1
%%%%%%%%%%%%%%%%%%%%%
	\item 
		Write a \textbf{function} that loops through a word and returns a list with every 
		other letter of the word starting with the \textbf{second} letter.
		The function will take a single argument $word$ (a string representing the word to process).

		\textbf{Examples:}		
		\begin{itemize}
			\item  skip\_letter(\csq{counterattack}) $\rightarrow$ 
				[\csq{o},\csq{n},\csq{e},\csq{a},\csq{t},\csq{c}]
			\item  skip\_letter(\csq{banana sunday}) $\rightarrow$
				[\csq{a},\csq{a},\csq{a},\csq{s},\csq{n},\csq{a}]
		\end{itemize}


%new_question
%%%%%%%%%%%%%%%%%%%%%
	% Problem 3
	% Difficulty: 1
%%%%%%%%%%%%%%%%%%%%%
	\item 
		Write a \textbf{function} that loops through and returns a list with every even number between two
		integers (inclusive). The arguments to the function will be $smaller\_num$ and 
		$larger\_num$.

		\textbf{Examples:}		
		\begin{itemize}
			\item  output\_even(37, 1050) $\rightarrow$ [38, 40, 42, \dots 1048, 1050], 
			\item  output\_even(1, 2000) $\rightarrow$ [2, 4, 6, \dots 1998, 2000], 
			\item  output\_even(50, 199) $\rightarrow$ [50, 52, 54, \dots 196, 198]
		\end{itemize}

%new_question
%%%%%%%%%%%%%%%%%%%%%
	% Problem 4
	% Difficulty: 1
%%%%%%%%%%%%%%%%%%%%%
	\item 
		Write a \textbf{function} that loops through and returns a list with every odd number between two
		integers (inclusive). The arguments to the function will be $smaller\_num$ and 
		$larger\_num$.

		\textbf{Examples:}		
		\begin{itemize}
			\item  output\_odd(37, 1050) $\rightarrow$ [37, 39, 41, \dots, 1049], 
			\item  output\_odd(1, 2000) $\rightarrow$ [1, 3, 5, \dots, 1999], 
			\item  output\_odd(50, 199) $\rightarrow$ [51, 53, 55, \dots, 197, 199]
		\end{itemize}


%end_of_questions
%make sure to leave at least one blank line below

%standard 7.2
%week 6


%start_of_questions



%new_question
%%%%%%%%%%%%%%%%%%%%%
	% Problem 5
	% Difficulty: 2
%%%%%%%%%%%%%%%%%%%%%
	\item 
		Given a positive integer $n$, the following rules will always create a sequence that 
		ends with 1, called Hailstone Sequence:
		\begin{enumerate}
			\item If $n$ is even, divide by 2
			\item If $n$ is odd, multiply by 3 and add 1 (i.e. $3n+1$)
			\item Continue until $n$ is 1
		\end{enumerate}
		Write a \textbf{function} that returns a list with the hailstone sequence starting at $n$. 
		The argument to the function will be $n$ (the integer to start the sequence from).
		\textbf{Examples:}		
		\begin{itemize}
			\item  hailstone\_seq(25) $\rightarrow$ [25, 76, 38, 19, 58 ... 8, 4, 2, 1], 
			\item  hailstone\_seq(40) $\rightarrow$ [40, 20, 10, 5, 16, 8, 4, 2, 1]
		\end{itemize}



%new_question
%%%%%%%%%%%%%%%%%%%%%
	% Problem 6
	% Difficulty: 2
%%%%%%%%%%%%%%%%%%%%%
	\item 
		%https://edabit.com/challenge/6Pf5GGG6HnzbB95gf
		Write a \textbf{function} that returns a list with the factors of a given integer. The argument of the function
		will be $num$ (integer to find factors for).

		\textbf{Examples:}		
		\begin{itemize}
			\item  find\_factors(12) $\rightarrow$ [1, 2, 3, 4, 6, 12], 
			\item  find\_factors(17) $\rightarrow$ [1, 17],
			\item  find\_factors(36) $\rightarrow$ [1, 2, 3, 4, 6, 9, 12, 18, 36]
		\end{itemize}



%new_question
%%%%%%%%%%%%%%%%%%%%%
	% Problem 7
	% Difficulty: 2
%%%%%%%%%%%%%%%%%%%%%
	\item
		Write a \textbf{function} that takes 3 numbers as arguments, $num\_1$ (first number), 
		$num\_2$ (second number), and $num\_3$ (third number). 
		Return a list of the integers in ascending order. 
		You may \textbf{not} use the built-in functions \textit{max}(), \textit{min}(), 
		\textit{sort}(), or \textit{sorted}().
		
	\textbf{Examples:}
	\begin{itemize}
		\item  ascending\_order(2, 3, 1) $\rightarrow$ [1, 2, 3], 
		\item  ascending\_order(10, 1, 25) $\rightarrow$ [1, 10, 25], 
		\item  ascending\_order(2, 45, 4) $\rightarrow$ [2, 4, 45] 
	\end{itemize}


%new_question
%%%%%%%%%%%%%%%%%%%%%
	% Problem 8
	% Difficulty: 2
%%%%%%%%%%%%%%%%%%%%%
	\item
		Write a \textbf{function} that takes 3 numbers as arguments, $num\_1$ (first number), 
		$num\_2$ (second number), and $num\_3$ (third number). 
		Return a list of the integers in descending order. 
		You may \textbf{not} use the built-in functions \textit{max}(), \textit{min}(), 
		\textit{sort}(), or \textit{sorted}().
		
	\textbf{Examples:}
	\begin{itemize}
		\item  descending\_order(2, 3, 1) $\rightarrow$ [3, 2, 1], 
		\item  descending\_order(10, 1, 25) $\rightarrow$ [25, 10, 1], 
		\item  descending\_order(2, 45, 4) $\rightarrow$ [45, 4, 2] 
	\end{itemize}



%new_question
%%%%%%%%%%%%%%%%%%%%%
	% Problem 11
	% Difficulty: 2
%%%%%%%%%%%%%%%%%%%%%
%https://edabit.com/challenge/izfXy5SGfeekmKExH
	\item 
		Write a function named \textit{add\_lists} that takes two lists $lyst1$ and $lyst2$ and adds the 
		first element in $lyst1$ with the first element in $lyst2$, the second element $lyst1$
		with the second element $lyst2$, etc. Return a new list containing the corresponding 
		sums of the list1 and list2.  You may assume both lists have the same length.
		%Return True if all element combinations add up to the same number. Otherwise, return False.

		\textbf{Examples:}		
		\begin{itemize}
			\item  add\_lists([1, 3, 3, 1], [4, 3, 6, 1]) $\rightarrow$ [5, 6, 8, 2] 
				( since 1+4=5; 3+3=6; 3+6=9; 1+1=2)
			\item  add\_lists([1, 8, 5, 0, -7], [0, -7, 4, 2, -6]) $\rightarrow$ [1, 1, 9, 2, -13]
			\item  add\_lists([1, 2], [-1, 1]) $\rightarrow$ [0, 3]
		\end{itemize}


%new_question
%%%%%%%%%%%%%%%%%%%%%
	% Problem 12
	% Difficulty: 2
%%%%%%%%%%%%%%%%%%%%%
%https://edabit.com/challenge/ksZrMdraPqHjvbaE6
	\item 
		Write a \textbf{function} that finds the largest even number in a list $numbers$. Return -1 if not found. 
		You may \textbf{not} use the built-in functions \textit{max}(), \textit{min}(), \textit{sort}(), or \textit{sorted}().

		\textbf{Examples:}		
		\begin{itemize}
			\item  largest\_even([3, 7, 2, 1, 7, 9, 10, 13]) $\rightarrow$ 10,
			\item  largest\_even([1, 3, 5, 7]) $\rightarrow$ -1,
			\item  largest\_even([0, 19, 18973623]) $\rightarrow$ 0
		\end{itemize}

%new_question
%%%%%%%%%%%%%%%%%%%%%
	% Problem 13
	% Difficulty: 2
%%%%%%%%%%%%%%%%%%%%%
	\item 
		Write a \textbf{function} that finds the largest odd number in a list $numbers$. Return -1 if not found. 
		You may \textbf{not} use the built-in functions \textit{max}(), \textit{min}(), \textit{sort}(), or \textit{sorted}().

		\textbf{Examples:}		
		\begin{itemize}
			\item  largest\_odd([3, 7, 2, 1, 7, 9, 10, 13]) $\rightarrow$ 13,
			\item  largest\_odd([2, 4, 6, 8]) $\rightarrow$ -1,
			\item  largest\_odd([0, 19, 18973623]) $\rightarrow$ 18973623
		\end{itemize}


%new_question
%%%%%%%%%%%%%%%%%%%%%
	% Problem 16
	% Difficulty: 2
%%%%%%%%%%%%%%%%%%%%%
%https://edabit.com/challenge/egMp3GWyN8TPptbZA
	\item
		YouTube currently displays a like and a dislike button, allowing you to express your opinions about particular content. 
		It's set up in such a way that you cannot like and dislike a video at the same time.
		There are two other interesting rules to be noted about the interface:
		\begin{enumerate}
			\item Pressing a button, which is already active, will undo your press.
			\item If you press the like button after pressing the dislike button, the like button overwrites the previous \csq{dislike} state. The same is true for the other way round.
		\end{enumerate}
		Write a \textbf{function} that takes in a list of button inputs $events$ and returns the final state.

		\textbf{Examples:}		
		\begin{itemize}
			\item  like\_or\_dislike([\csq{dislike}]) $\rightarrow$ \csq{dislike},
			\item  like\_or\_dislike([\csq{like}, \csq{like}]) $\rightarrow$ \csq{nothing},
			\item  like\_or\_dislike([\csq{dislike}, \csq{like}]) $\rightarrow$ \csq{like},
			\item  like\_or\_dislike([\csq{like}, \csq{dislike}, \csq{dislike}]) $\rightarrow$ \csq{nothing},
		\end{itemize}




%new_question
%%%%%%%%%%%%%%%%%%%%%
	% Problem 17
	% Difficulty: 2
%%%%%%%%%%%%%%%%%%%%%
%https://edabit.com/challenge/jwzgYjymYK7Gmro93
	\item
		Write a \textbf{function} that takes two arguments, a list and an item.  The function should return the indices of all occurrences of the $item$ in the list.

		\textbf{Examples:}		
		\begin{itemize}
			\item  get\_indices( [1, 5, 5, 2, 7], 7) $\rightarrow$ [4]
			\item  get\_indices( [1, 5, 5, 2, 7], 5) $\rightarrow$ [1, 2]
			\item  get\_indices( [1, 5, 5, 2, 7], 8) $\rightarrow$ []
			\item  get\_indices( [\csq{a}, \csq{a}, \csq{b}, \csq{a}, \csq{b}, \csq{a}], \csq{a}) 
				$\rightarrow$ [0, 1, 3, 5]
		\end{itemize}


%new_question
%%%%%%%%%%%%%%%%%%%%%
	% Problem 18
	% Difficulty: 2
%%%%%%%%%%%%%%%%%%%%%
%https://edabit.com/challenge/BuwHwPvt92yw574zB
	\item
		Write a \textbf{function} that takes two numbers as arguments $num$ and $length$ and returns a 
		list of multiples of $num$ until the list length reaches $length$.

		\textbf{Examples:}		
		\begin{itemize}
			\item  list\_of\_multiples(7, 5) $\rightarrow$ [7, 14, 21, 28, 35]
			\item  list\_of\_multiples(12, 10) $\rightarrow$ [12, 24, 36, 48, 60, 72, 84, 96, 108, 120]
			\item  list\_of\_multiples(17, 6) $\rightarrow$ [17, 34, 51, 68, 85, 102]
		\end{itemize}

%end_of_questions
%make sure to leave at least one blank line below


%standard 7.3
%week6

%start_of_questions



%new_question
%%%%%%%%%%%%%%%%%%%%%
	% Problem 9
	% Difficulty: 3
%%%%%%%%%%%%%%%%%%%%%
%https://edabit.com/challenge/hYiCzkLBBQSeF8fKF
	\item 
		In BlackJack, cards are counted with -1, 0, 1 values:
		\begin{itemize}
			\item 2, 3, 4, 5, 6 are counted as +1
			\item 7, 8, 9 are counted as 0
			\item 10, j, q, k, a are counted as -1
		\end{itemize}
		Write a \textbf{function} that takes a list called $cards$, counts the number, 
		and returns it from the list of cards provided.

		\textbf{Examples:}		
		\begin{itemize}
			\item  count([5, 9, 10, 3, \csq{j}, \csq{a}, 4, 8, 5]) $\rightarrow$ 1, 
			\item  count([\csq{a}, \csq{a}, \csq{k}, \csq{q}, \csq{q}, \csq{j}]) $\rightarrow$ -6, 
			\item  count([\csq{a}, 5, 5, 2, 6, 2, 3, 8, 9, 7]) $\rightarrow$ 5
		\end{itemize}



%new_question
%%%%%%%%%%%%%%%%%%%%%
	% Problem 10
	% Difficulty: 3
%%%%%%%%%%%%%%%%%%%%%
%https://edabit.com/challenge/rMr8yRxS8TeF9pDyn
	\item 
		There's a great war between the even and odd numbers. Many numbers already lost 
		their lives in this war and it's your task to end this. You have to determine 
		which group sums larger: the evens or the odds. The larger group wins.

		Write a \textbf{function} that takes a list of integers named \textit{numbers}, 
		sums the even numbers and odd numbers separately, then returns which of the two
		sums is larger.

		\textbf{Examples:}		
		\begin{itemize}
			\item  war\_of\_numbers([2, 8, 7, 5]) $\rightarrow$ 
				\csq{odds}, (since 2 + 8 = 10, 7 + 5 = 12, odds is larger) 
			\item  war\_of\_numbers([12, 90, 75, 1, 1]) $\rightarrow$ \csq{evens}, 
				(12 + 90 = 102, 75 + 1 + 1 = 77, evens is larger) 
			\item  war\_of\_numbers([2, 10, 22, 243]) $\rightarrow$ \csq{odds}
		\end{itemize}



%new_question
%%%%%%%%%%%%%%%%%%%%%
	% Problem 14
	% Difficulty: 3
%%%%%%%%%%%%%%%%%%%%%
%https://edabit.com/challenge/2yHQwkecEHZBfHcmN
	\item
		To train for an upcoming marathon, Johnny goes on one long-distance run each Saturday. 
		He wants to track how often the number of miles he runs exceeds the previous Saturday. 
		This is called a progress day. Write a \textbf{function} that takes in a list of miles 
		run every Saturday and returns Johnny's total number of progress days.

		\textbf{Examples:}		
		\begin{itemize}
			\item  progress\_days([3, 4, 1, 2]) $\rightarrow$ 2, 
				(Two progress days, day 2 since $(4>3)$ and day 4 since $(2>1)$)
			\item  progress\_days([10, 11, 12, 9, 10]) $\rightarrow$ 3, 
			\item  progress\_days([6, 5, 4, 3, 2, 9]) $\rightarrow$ 1, 
			\item  progress\_days([9, 9]) $\rightarrow$ 0
		\end{itemize}

%new_question
%%%%%%%%%%%%%%%%%%%%%
	% Problem 15
	% Difficulty: 3
%%%%%%%%%%%%%%%%%%%%%
	\item
		To train for an upcoming marathon, Samuel goes on one long-distance run each Saturday. 
		He wants to track how often the number of miles he runs fall short of the previous Saturday. 
		This is called a lag day. Write a \textbf{function} that takes in a list of miles 
		run every Saturday and returns Samuel's total number of lag days.

		\textbf{Examples:}		
		\begin{itemize}
			\item  lag\_days([5, 3, 2, 1]) $\rightarrow$ 3, 
				(3 lag days, day2 since (3$<$5), day3 since (2$<$3), and day4 since (1$<$2))
			\item  lag\_days([10, 11, 12, 9, 10]) $\rightarrow$ 1, 
			\item  lag\_days([6, 5, 4, 3, 2, 9]) $\rightarrow$ 4, 
			\item  lag\_days([9, 9]) $\rightarrow$ 0
		\end{itemize}

%new_question
%%%%%%%%%%%%%%%%%%%%%
	% Problem 19
	% Difficulty: 3
%%%%%%%%%%%%%%%%%%%%%%https://leetcode.com/problems/check-if-a-string-is-an-acronym-of-words/description/
	\item 
		Let $\bm{s}$ be a string and $\bm{words}$ be a list of strings. The string $\bm{s}$ is 
		considered an acronym of $\bm{words}$ if it can be formed by concatenating the first 
		character of each string in $\bm{words}$ in order. For example, \texttt{"ab"} can be formed 
		from [\texttt{"apple"}, \texttt{"banana"}], but it can't be formed from [\texttt{"bear"}, 
		\texttt{"aardvark"}]. Write a function that takes a string $\bm{s}$ and a list of strings $
		\bm{words}$, and returns True if $\bm{s}$ is an acronym of $\bm{words}$, and False 
		otherwise.

		\textbf{Examples:}		
		\begin{itemize}
			\item  is\_acronym(\texttt{"abc"}, [\texttt{"alice"}, \texttt{"bob"}, \texttt{"charlie"}] ) 
				$\rightarrow$ True\\
				Explanation: The first character in the words \texttt{"alice"}, \texttt{"bob"}, and 
				\texttt{"charlie"} are \texttt{"a"}, \texttt{"b"}, and \texttt{"c"}, respectively. 
				Hence, s = \texttt{"abc"} is the acronym. 			
			\item  is\_acronym(\texttt{"a"}, [\texttt{"an"}, \texttt{"apple"}] ) $\rightarrow$ False\\
				Explanation: The first character in the words \texttt{"an"} and \texttt{"apple"} are 
				\texttt{"a"} and \texttt{"a"}, respectively. The acronym formed by concatenating these 
				characters is \texttt{"aa"}. Hence, s = \texttt{"a"} is not the acronym.
			\item  is\_acronym(\texttt{"ngguoy"}, [\texttt{"never"}, \texttt{"gonna"}, \texttt{"give"}, 
				\texttt{"up"}, \texttt{"on"}, \texttt{"you"}]) $\rightarrow$ True\\\
				Explanation: By concatenating the first character of the words in the array, we get the 
				string \texttt{"ngguoy"}. Hence, s = \texttt{"ngguoy"} is the acronym.
			\item  is\_acronym(\texttt{"ab"}, [\texttt{"apple"}, \texttt{"banana"}, \texttt{cat}]) 
				$\rightarrow$ False\\
				Explanation: Wrong length
			\item  is\_acronym(\texttt{"ab"}, [\texttt{"apple"}, \texttt{""}, \texttt{cat}]) 
				$\rightarrow$ False\\
				Explanation: Can't get the first letter in the second string of $\bm{words}$.
	
		\end{itemize}

	%Explanation: By concatenating the first character of the words in the array, we get the string "ngguoy". 
	%Hence, s = "ngguoy" is the acronym.


%end_of_questions
%make sure to leave at least one blank line below






%end_of_questions




%https://leetcode.com/problems/two-sum/description/?envType=problem-list-v2&envId=hash-table




\end{enumerate}
\end{document}

















%new_question
%%%%%%%%%%%%%%%%%%%%%
	% Problem 1
	% Difficulty: 1
%%%%%%%%%%%%%%%%%%%%%
%https://edabit.com/challenge/R4D59C9CQbJvqWaKd
	%new
	\item
		A baseball player's batting average is calculated by the following formula:\\
		\begin{center}
		BA = (number of hits) / (number of official at-bats)\\
		\end{center}
		Batting averages are always expressed rounded to the nearest thousandth with no leading zero. 
		The top 3 MLB batting averages of all-time are:
		\begin{enumerate}
			\item Ty Cobb .366
			\item Rogers Hornsby .358
			\item Shoeless Joe Jackson .356
		\end{enumerate}
		The given list represents a season of games. Each list item indicates a player's [hits, official at bats] per game. 
		Write a \textbf{function} that takes in a list of $games$ (2 dimensional list) and returns a string with the player's
		 seasonal batting average rounded to the nearest thousandth.

		\textbf{Examples:}		
		\begin{itemize}
			\item  batting\_avg([[0, 0], [1, 3], [2, 2], [0, 4], [1, 5]]) $\rightarrow$ \csq{.286},
			\item  batting\_avg([[2, 5], [2, 3], [0, 3], [1, 5], [2, 4]]) $\rightarrow$ \csq{.350},
			\item  batting\_avg([[2, 3], [1, 5], [2, 4], [1, 5], [0, 5]]) $\rightarrow$ \csq{.273},
		\end{itemize}

%new_question
%%%%%%%%%%%%%%%%%%%%%
	% Problem 1
	% Difficulty: 1
%%%%%%%%%%%%%%%%%%%%%
%https://edabit.com/challenge/J2apiSnJE4RaGTj6x
	%new
	\item
		Given what is supposed to be typed $correct\_phrase$ and what is actually typed $actual\_phrase$, write a \textbf{function} that returns the broken key(s).
		 The function looks like:\\
		\begin{center}
		find\_broken\_keys(correct phrase, what you actually typed)\\
		\end{center}

		\textbf{Examples:}		
		\begin{itemize}
			\item  find\_broken\_keys(\csq{happy birthday}, \csq{hawwy birthday}) $\rightarrow$ [\csq{p}],
			\item  find\_broken\_keys(\csq{starry night}, \csq{starrq light}) $\rightarrow$ [\csq{y}, \csq{n}]
			\item  find\_broken\_keys(\csq{beethoven}, \csq{affthoif5}) $\rightarrow$ [\csq{b}, \csq{e}, \csq{v}, \csq{n}]
		\end{itemize}



%new_question
%%%%%%%%%%%%%%%%%%%%%
	% Problem 1
	% Difficulty: 1
%%%%%%%%%%%%%%%%%%%%%
%https://edabit.com/challenge/MFteyMABeuGaga3a7
	%new
	\item
		When coloring a striped pattern, you may start by coloring each square sequentially, meaning you spend time needing to switch coloring pencils.
		Write a \textbf{function} where given a list of colors $cols$, return how long it takes to color the whole pattern. Note the following times:
		\begin{itemize}
			\item  It takes 1 second to switch between pencils.
			\item  It takes 2 seconds to color a square.
		\end{itemize}

		\textbf{Examples:}		
		\begin{itemize}
			\item  color\_pattern\_times([\csq{Red}, \csq{Blue}, \csq{Red}, \csq{Blue}, \csq{Red}]) $\rightarrow$ 14, Since there are 5 colors with 2 seconds each and you need to switch colors pencils 4 times. 
			so (2 x 5) + (4 x 1) = 14
			\item  color\_pattern\_times([\csq{Blue}]) $\rightarrow$ 2,
			\item  color\_pattern\_times([\csq{Red}, \csq{Yellow}, \csq{Green}, \csq{Blue}]) $\rightarrow$ 11,
			\item  color\_pattern\_times([\csq{Blue}, \csq{Blue}, \csq{Blue}, \csq{Red}, \csq{Red}, \csq{Red}]) $\rightarrow$ 13
		\end{itemize}


%new_question
%%%%%%%%%%%%%%%%%%%%%
	% Problem 1
	% Difficulty: 1
%%%%%%%%%%%%%%%%%%%%%
	%new
	\item
		When painting a row of fence posts, you must change brushes whenever the color changes.
		Write a \textbf{function} where given a list of colors $cols$, returns how long it takes to paint the entire fence. Consider the following times:
		\begin{itemize}
			\item  It takes 3 seconds to paint one fence post.
			\item  It takes 2 seconds to switch brushes when changing colors.
		\end{itemize}

		\textbf{Examples:}		
		\begin{itemize}
			\item  paint\_fence\_time([\csq{Red}, \csq{Blue}, \csq{Red}, \csq{Blue}, \csq{Red}]) $\rightarrow$ 19, Since there are 5 fence posts with 3 seconds each, and you switch brushes 4 times.
			So, (3 × 5) + (2 × 4) = 19
			\item  paint\_fence\_time([\csq{Blue}]) $\rightarrow$ 3
			\item  paint\_fence\_time([\csq{Red}, \csq{Yellow}, \csq{Green}, \csq{Blue}]) $\rightarrow$ 17
			\item  paint\_fence\_time([\csq{Blue}, \csq{Blue}, \csq{Blue}, \csq{Red}, \csq{Red}, \csq{Red}]) $\rightarrow$ 17
		\end{itemize}


%new_question
%%%%%%%%%%%%%%%%%%%%%
	% Problem 1
	% Difficulty: 1
%%%%%%%%%%%%%%%%%%%%%
%https://edabit.com/challenge/q7BdzRw4j7zFfFb4R
	%new
	\item
		Write a \textbf{function} that takes two lists $letters$ and $numbers$ and combines them by alternatingly taking elements from each list in turn.
		\begin{itemize}
			\item  The lists may be of different lengths, with at least one character / digit.
			\item  The first list will contain string characters (lowercase, a-z).
			\item  The second list will contain integers (all positive).
		\end{itemize}

		\textbf{Examples:}		
		\begin{itemize}
			\item  merge\_lists([\csq{a}, \csq{b}, \csq{c}, \csq{d}, \csq{e}], [1, 2, 3, 4, 5]) $\rightarrow$ [\csq{a}, 1, \csq{b}, 2, \csq{c}, 3, \csq{d}, 4, \csq{e}, 5]
			\item  merge\_lists([\csq{a}, \csq{b}, \csq{c}, \csq{d}, \csq{e}, \csq{f}], [1, 2, 3]) $\rightarrow$ [\csq{a}, 1, \csq{b}, 2, \csq{c}, 3, \csq{d}, \csq{e}, \csq{f}]
			\item  merge\_lists([\csq{f}, \csq{d}, \csq{w}, \csq{t}], [5, 3, 7, 8]) $\rightarrow$ [\csq{f}, 5, \csq{d}, 3, \csq{w}, 7, \csq{t}, 8]
		\end{itemize}


%new_question
%%%%%%%%%%%%%%%%%%%%%
	% Problem 1
	% Difficulty: 1
%%%%%%%%%%%%%%%%%%%%%
	%new
	\item
		Write a \textbf{function} that takes two lists, $words$ and $symbols$, and merges them by alternating elements from each list in turn.
		\begin{itemize}
			\item  The lists may be of different lengths, but at least one element is guaranteed.
			\item  The first list $words$ will contain strings (lowercase words).
			\item  The second list (symbols) will contain single-character symbols (e.g., !, @, \#, \$).
		\end{itemize}

		\textbf{Examples:}		
		\begin{itemize}
			\item  merge\_words\_symbols([\csq{apple}, \csq{banana}, \csq{cherry}], [\csq{!}, \csq{@}, \csq{\#}]) $\rightarrow$ [\csq{apple}, \csq{!}, \csq{banana}, \csq{@}, \csq{cherry}, \csq{\#}]
			\item  merge\_words\_symbols([\csq{hello}, \csq{world}], [\csq{\$}, \csq{\%}, \csq{\&}, \csq{*}]) $\rightarrow$ [\csq{hello}, \csq{\$}, \csq{world}, \csq{\%}, \csq{\&}, \csq{*}]
			\item  merge\_words\_symbols([\csq{cat}, \csq{dog}, \csq{elephant}],[\csq{\^}, \csq{\&}]) $\rightarrow$ [\csq{cat}, \csq{\^}, \csq{dog}, \csq{\&}, \csq{elephant}]
		\end{itemize}


%new_question
%%%%%%%%%%%%%%%%%%%%%
	% Problem 1
	% Difficulty: 1
%%%%%%%%%%%%%%%%%%%%%
%https://edabit.com/challenge/5uMJmbN2uihcyEu75
	%new
	\item
		Write a \textbf{function} that takes a list of $hours$ and returns the total weekly salary.
		\begin{itemize}
			\item  The input list hours is listed sequentially, ordered from Monday to Sunday.
			\item  A worker earns \$10 an hour for the first 8 hours.
			\item  For every overtime hour, he earns \$15.
			\item  On weekends, the employer pays double the usual rate, regardless how many hours were worked previously that week. 
			For instance, 10 hours worked on a weekday would pay 80+30 = \$110, but on a weekend it would pay 160+60 = \$220.
		\end{itemize}

		\textbf{Examples:}		
		\begin{itemize}
			\item  weekly\_pay([8, 8, 8, 8, 8, 0, 0]) $\rightarrow$ 400
			\item  weekly\_pay([10, 10, 10, 0, 8, 0, 0]) $\rightarrow$ 410
			\item  weekly\_pay([0, 0, 0, 0, 0, 12, 0]) $\rightarrow$ 280
		\end{itemize}


%new_question
%%%%%%%%%%%%%%%%%%%%%
	% Problem 1
	% Difficulty: 1
%%%%%%%%%%%%%%%%%%%%%
%https://edabit.com/challenge/5uMJmbN2uihcyEu75
	%new
	\item
		Write a \textbf{function} that retrieves every number that is strictly larger than every number that follows it.

		\textbf{Examples:}		
		\begin{itemize}
			\item  larger\_than\_right([3, 13, 11, 2, 1, 9, 5]) $\rightarrow$ [13, 11, 9, 5]
			\item  larger\_than\_right([5, 5, 5, 5, 5, 5]) $\rightarrow$ [5], Since the last number is larger than all the numbers on the right
			\item  larger\_than\_right([5, 9, 8, 7]) $\rightarrow$ [9, 8, 7]
		\end{itemize}

















%new_question
%%%%%%%%%%%%%%%%%%%%%
	% Problem 1
	% Difficulty: 1
%%%%%%%%%%%%%%%%%%%%%
	%paper-based
	\item
		You are in charge of the cake for a child's birthday. You have decided the cake will have one candle for each year of their total
		age. They will only be able to blow out the tallest of the candles.\\
		Count how many candles are tallest.

		\textbf{Examples:}		
		\begin{itemize}
			\item  birthday\_cake\_candles([4, 4, 1, 3]) $\rightarrow$ 2
				\# The maximum height of candles are four units high
				\# There are two of them, so you return 2
			\item  birthday\_cake\_candles([3, 2, 1, 3]) $\rightarrow$ 2
			\item  birthday\_cake\_candles([82, 49, 82, 82, 41, 82, 15, 63, 38, 25]) $\rightarrow$ 4
		\end{itemize}

%new_question
%%%%%%%%%%%%%%%%%%%%%
	% Problem 1
	% Difficulty: 1
%%%%%%%%%%%%%%%%%%%%%
	%paper-based
	\item
		A group of friends have decided to start a secret society. The name will be the first letter of each of their names.
		Create a function that takes in a list of names and returns the name of the secret society.

		\textbf{Examples:}		
		\begin{itemize}
			\item  society\_name(["Adam", "Sarah", "Malcom"]) $\rightarrow$ "ASM"
			\item  society\_name(["Cho", "Harry", "Luna", "Newt"]) $\rightarrow$ "CHLN"
			\item  society\_name(["Phoebe", "Chandler", "Rachel", "Ross", "Monica", "Joey"]) $\rightarrow$ "PCRRMJ"
		\end{itemize}


%new_question
%%%%%%%%%%%%%%%%%%%%%
	% Problem 1
	% Difficulty: 1
%%%%%%%%%%%%%%%%%%%%%
	%paper-based
	\item
		Given a non-empty list of integers, write code that prints the minimum value in the list without using the built-in function min().

		\textbf{Examples:}		
		\begin{itemize}
			\item  lyst = [5, 12, 3, 8, 4] $\rightarrow$ 3
			\item  lyst = [78, 121, 312, 89, 1221] $\rightarrow$ 78
			\item  lyst = [9, 0, -1, 8] $\rightarrow$ -1
		\end{itemize}


%new_question
%%%%%%%%%%%%%%%%%%%%%
	% Problem 1
	% Difficulty: 1
%%%%%%%%%%%%%%%%%%%%%
	%paper-based
	\item
		Write a function named $find\_even\_nums$ that takes a single number as an argument and returns a list containing all
		the even numbers between 1 and the provided arugment (inclusively).

		\textbf{Examples:}		
		\begin{itemize}
			\item  find\_even\_nums(8) $\rightarrow$ [2, 4, 6, 8]
			\item  find\_even\_nums(4) $\rightarrow$ [2, 4]
			\item  find\_even\_nums(2) $\rightarrow$ [2]
		\end{itemize}



%new_question
%%%%%%%%%%%%%%%%%%%%%
	% Problem 1
	% Difficulty: 1
%%%%%%%%%%%%%%%%%%%%%
	%paper-based
	\item 
		Create an initially empty list named $fruit\_lyst$. Write code to add the following fruit to the $fruit\_lyst$ one at a time:
		"apple", "banana", "orange".
		The result should be ["apple", "banana", "orange"]

%new_question
%%%%%%%%%%%%%%%%%%%%%
	% Problem 1
	% Difficulty: 1
%%%%%%%%%%%%%%%%%%%%%
	%paper-based
	\item 
		Create a list of 100 integers whose value and index are the same, for example L[5] = 5.


%new_question
%%%%%%%%%%%%%%%%%%%%%
	% Problem 1
	% Difficulty: 1
%%%%%%%%%%%%%%%%%%%%%
	%paper-based
	\item 
		Write code that calculates the mean of a list of integers names $numbers$


%new_question
%%%%%%%%%%%%%%%%%%%%%
	% Problem 1
	% Difficulty: 1
%%%%%%%%%%%%%%%%%%%%%
	%paper-based
	\item 
		Given a list of positive integers named $my\_list$, write code that prints the largest number in $my\_list$, without using the
		built-in function max().

%new_question
%%%%%%%%%%%%%%%%%%%%%
	% Problem 1
	% Difficulty: 1
%%%%%%%%%%%%%%%%%%%%%
	%paper-based
	\item 
		For a list of distinct integers named $numbers$, write code to print any paris that add to 10.
		For example, if numbers = [3, 8, -1, 5, 2, 7], then it should print 3 \& 7 and 2 \& 8.
		Hint: Distinct mean there are no repeated numbers in the list.


%new_question
%%%%%%%%%%%%%%%%%%%%%
	% Problem 1
	% Difficulty: 1
%%%%%%%%%%%%%%%%%%%%%
	%paper-based
	\item 
		Create a list of all the integers between 1 and 1000 that are multiples of 8 and 12. 
		Hint: The first five elements of this list should be 24, 48, 72, 96, 120

%new_question
%%%%%%%%%%%%%%%%%%%%%
	% Problem 1
	% Difficulty: 1
%%%%%%%%%%%%%%%%%%%%%
	%paper-based
	\item 
		Write a function named $list\_of\_squares$ that takes a list of numbers as an argument, and
		returns a list of the squares of those numbers. For example, [1, 2, 3] $\rightarrow$ [1, 4, 9]

%new_question
%%%%%%%%%%%%%%%%%%%%%
	% Problem 1
	% Difficulty: 1
%%%%%%%%%%%%%%%%%%%%%
	%paper-based
	\item 
		Write code to create a list containing 10,000 x's. The result should look like [x, x,..., x]

%new_question
%%%%%%%%%%%%%%%%%%%%%
	% Problem 1
	% Difficulty: 1
%%%%%%%%%%%%%%%%%%%%%
	%paper-based
	\item 
		Write a function that takes a list of positive numbers as an argument, and returns the maximum value of the list.


%new_question
%%%%%%%%%%%%%%%%%%%%%
	% Problem 1
	% Difficulty: 1
%%%%%%%%%%%%%%%%%%%%%
	%paper-based
	\item 
		For a list of integers named $data$, write code to store all the odd numbers. Store them in a list named $odd\_data$.

%new_question
%%%%%%%%%%%%%%%%%%%%%
	% Problem 1
	% Difficulty: 1
%%%%%%%%%%%%%%%%%%%%%
	%paper-based
	\item 
		For a list of floats named $data$, write code that multiplies each number by 2.

%new_question
%%%%%%%%%%%%%%%%%%%%%
	% Problem 1
	% Difficulty: 1
%%%%%%%%%%%%%%%%%%%%%
	%paper-based
	\item 
		Given a list of words named $myList$, print all the words from $myList$ that start with either an M or m.


%new_question
%%%%%%%%%%%%%%%%%%%%%
	% Problem 1
	% Difficulty: 1
%%%%%%%%%%%%%%%%%%%%%
	%paper-based
	\item 
		Write a loop that replaces each number in a list named \textbf{numbers} with its absolute value.

%new_question
%%%%%%%%%%%%%%%%%%%%%
	% Problem 1
	% Difficulty: 1
%%%%%%%%%%%%%%%%%%%%%
	%paper-based
	\item 
		The range of a list of numbers is the largest number in the list minus the smallest. Given a list
		of positive numbers named $data$, write code to calculate the range of the list. Do not use the built-in functions
		min() and max().


%new_question
%%%%%%%%%%%%%%%%%%%%%
	% Problem 1
	% Difficulty: 1
%%%%%%%%%%%%%%%%%%%%%
	%paper-based
	\item 
		Create a list that stores every other odd number starting at 1. Continue storing numbers in this way until you
		have stored a total of 100 numbers.\\
		Hint: the result should look like 1, 5, 8, 13...

%new_question
%%%%%%%%%%%%%%%%%%%%%
	% Problem 1
	% Difficulty: 1
%%%%%%%%%%%%%%%%%%%%%
	%paper-based
	\item 
		Given a list of words named $words$, print all the words from $words$ that contain an x.
		That is, if a word has an $x$ in it, print that word. Additionally, report how many words are printed in total.


%new_question
%%%%%%%%%%%%%%%%%%%%%
	% Problem 1
	% Difficulty: 1
%%%%%%%%%%%%%%%%%%%%%
	%paper-based
	\item 
		Create a list containing exactly 1000 a's.


%new_question
%%%%%%%%%%%%%%%%%%%%%
	% Problem 1
	% Difficulty: 1
%%%%%%%%%%%%%%%%%%%%%
	%paper-based
	\item 
		Given the list letters = ["a", "b", "c", ..., "z"], create a new list that is ["a1", "b2", "c3", ..., "z26"]

%new_question
%%%%%%%%%%%%%%%%%%%%%
	% Problem 1
	% Difficulty: 1
%%%%%%%%%%%%%%%%%%%%%
	%paper-based
	\item 
		For a list of integers named myList, write code to store the numbers from myList that are divisble by 5 and 8. 
		Store them in a list named newList.

%new_question
%%%%%%%%%%%%%%%%%%%%%
	% Problem 1
	% Difficulty: 1
%%%%%%%%%%%%%%%%%%%%%
	%paper-based
	\item 
		For a list of distinct integers named numbers, write code to print any pairs that add to 10. For example, if
		numbers = [3, 8, -1, 5, 2, 7], then it should print 3 \& 7 and 2 \& 8.
		Hint: Distinct means there are no repeated numbers in the list.


%new_question
%%%%%%%%%%%%%%%%%%%%%
	% Problem 1
	% Difficulty: 1
%%%%%%%%%%%%%%%%%%%%%
	%new
	\item 
		Write a \textbf{function} that returns the first $n$ numbers in the Fibonacci sequence. The argument of the function will be $n$ (the number of Fibonacci numbers to
		 generate). The Fibonacci sequence is defined as follows:
		\begin{itemize}
			\item  For n = 0, F(0) = 0
			\item  For n = 1, F(1) = 1
			\item  For $n \geq 2$, F(n) = F(n-1) + F(n-2)
		\end{itemize}

		\textbf{Examples:}		
		\begin{itemize}
			\item  fibonacci(5) $\rightarrow$ [0, 1, 1, 2, 3], 
			\item  fibonacci(7) $\rightarrow$ [0, 1, 1, 2, 3, 5, 8], 
			\item  fibonacci(10) $\rightarrow$ [0, 1, 1, 2, 3, 5, 8, 13, 21, 34]
		\end{itemize}


%new_question
%%%%%%%%%%%%%%%%%%%%%
	% Problem 1
	% Difficulty: 1
%%%%%%%%%%%%%%%%%%%%%
%https://edabit.com/challenge/Yfm3h3nT3apARd4gC
	%new
	\item 
		Write a \textbf{function} that takes a list called $dice\_rolls$ consisting of dice rolls from 1-6. Return the sum of your rolls with the following conditions:
		\begin{enumerate}
			\item If a 1 is rolled, that is bad luck. The next roll counts as 0.
			\item If a 6 is rolled, that is good luck. The next roll is multiplied by 2.
			\item The list length will always be 3 or higher.
		\end{enumerate}
		\textbf{Examples:}		
		\begin{itemize}
			\item  rolls([1, 2, 3]) $\rightarrow$ 4, 
			\item  rolls([2, 6, 2, 5]) $\rightarrow$ 17, 
			\item  rolls([6, 1, 1]) $\rightarrow$ 8
		\end{itemize}


%new_question
%%%%%%%%%%%%%%%%%%%%%
	% Problem 1
	% Difficulty: 1
%%%%%%%%%%%%%%%%%%%%%
%https://edabit.com/challenge/hYiCzkLBBQSeF8fKF
	%new
	\item 
		A factor chain is a list where each previous element is a factor of the next consecutive element. The following is a factor chain:
		\begin{verbatim}
			[3, 6, 12, 36]
			# 3 is a factor of 6
			# 6 is a factor of 12
			# 12 is a factor of 36
		\end{verbatim}
		Write a \textbf{function} that takes a list called $factor\_chain$ and returns whether or not a list is a factor chain.

		\textbf{Examples:}		
		\begin{itemize}
			\item  is\_factor\_chain([1, 2, 4, 8, 16, 32]) $\rightarrow$ True, 
			\item  is\_factor\_chain([1, 1, 1, 1, 1, 1]) $\rightarrow$ True, 
			\item  is\_factor\_chain([2, 4, 6, 7, 12]) $\rightarrow$ False, 
			\item  is\_factor\_chain([10, 1]) $\rightarrow$ False, 
		\end{itemize}


%new_question
%%%%%%%%%%%%%%%%%%%%%
	% Problem 1
	% Difficulty: 1
%%%%%%%%%%%%%%%%%%%%%
	%new
	\item
		In a debate competition, each judge casts a vote for \csq{agree} or \csq{disagree}.
		\begin{enumerate}
			\item If a judge changes their vote, the new vote overwrites the previous one.
			\item If a judge votes the same way twice, they withdraw their vote.
			\item If all votes are withdrawn, the result is \csq{undecided}.
		\end{enumerate}
		Write a \textbf{function} that takes in a list of votes $votes$ and returns the final decision.

		\textbf{Examples:}		
		\begin{itemize}
			\item  judge\_vote([\csq{agree}]) $\rightarrow$ \csq{agree},
			\item  judge\_vote([\csq{disagree}, \csq{agree}]) $\rightarrow$ \csq{agree},
			\item  judge\_vote([\csq{agree}, \csq{agree}]) $\rightarrow$ \csq{undecided},
			\item  judge\_vote([\csq{agree}, \csq{disagree}, \csq{disagree}, \csq{agree}]) $\rightarrow$ \csq{undecided}
		\end{itemize}