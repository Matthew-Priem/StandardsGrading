\documentclass{article}

\usepackage{amsmath}
\usepackage{amsfonts} % For math fonts.
\usepackage{amssymb} % For other math symbols not covered by amsmath.
\usepackage[pdftex]{graphicx} % For pictures, use %\includegraphics[scale=decimal]{pic.png}; must be a .png file type.
\usepackage{multicol}
\usepackage{textcomp}
\usepackage[colorlinks = true, urlcolor = blue]{hyperref}
\usepackage{enumitem}
\usepackage{graphbox} 
\usepackage{subfig}
\usepackage{multicol}

\newcommand{\tab}{\hspace*{0.25in}}

\usepackage{tikz}
\usetikzlibrary{positioning, calc}
\usetikzlibrary{shapes.geometric,angles,quotes}
\usepackage{tikz-3dplot}

\newcommand{\csq}[1]{\reflectbox{''}#1''}  %This produces CS style quotes.



\usepackage{fullpage}
\usepackage{listings}
\lstset
{ %Formatting for code in appendix
    language=Python,
    basicstyle=\footnotesize,
    numbers=left,
    stepnumber=1,
    showstringspaces=false,
    tabsize=2,
    breaklines=true,
    breakatwhitespace=false,
}


\begin{document}


\begin{flushright}
classes 2\end{flushright}

\vspace*{-1.5em}
\noindent\makebox[\linewidth]{\rule{\paperwidth}{0.4pt}}


\vspace*{2em}

\begin{enumerate}

%standard 12.1

%start_of_questions


%new_question

%%%%%%%%%%%%%%%%%%%%%
	% Problem 1
	% Difficulty: 1
%%%%%%%%%%%%%%%%%%%%%
	\item 
	\begin{enumerate}
		\item 
			Write a class for a \textit{Student} with the below instance variable and methods.\\
			A \textit{Student} should start (be initialized) with both a name and major.\\
			You may pick anything you like for the string representation of the object.
			\begin{flushright}
			\begin{tabular}{|l|} \hline
				Student\\ \hline
				name\\ major\\ \hline
				get\_major\\ set\_major\\ \_\_str\_\_ \\ \hline
			\end{tabular}
			\end{flushright}
		
		\item 
			Write a class for \textit{Course} here at MnSU.  The \textit{Course} should start 
			with a name and a number, but with no students. That is, it should 
			not have the ability to be initialized with students in it. However, the 
			\textit{Course} should have the ability to add students as well as show all of the
			students in the \textit{Course}.\\
			You may pick anything you like for the string representation of the object.
			\begin{flushright}
			\begin{tabular}{|l|} \hline 
				Course\\ \hline
				course\_name\\ course\_number\\ students\\ \hline
				get\_number\\ set\_number\\ add\_student\\ show\_student\_enrollment\\ 
					\_\_str\_\_ \\ \hline
			\end{tabular}
			\end{flushright}

		\item
			Create an instance of the \textit{Course} class and add 2 \textit{Student}s to it.
	\end{enumerate}
\pagebreak




%new_question

%%%%%%%%%%%%%%%%%%%%%
	% Problem 2
	% Difficulty: 1
%%%%%%%%%%%%%%%%%%%%%
	\item
	\begin{enumerate}
		\item
			Write a class for a Duck with the below instance variables and methods. \\ 
			The Duck object should have the ability to be passed both initial values.\\  
			You may pick anything you like for the string representation of the object.\\
			A Duck says, ``Quack.''\\  
			As part of your class, write a method called speak, that makes a duck quack!\\
			Hint: for this method, you can just print the word Quack!
			\begin{flushright}
			\begin{tabular}{|l|}
				\hline
				Duck\\ \hline
				name\\	color\\	 \hline
				get\_color \\ set\_color \\ speak \\ \_\_str\_\_ \\ \hline
			\end{tabular}
			\end{flushright}

		\item
			Write a class for a Pond. \\
			The class should start (instantiate) with a name, and no Ducks in it. \\ 
			Write a method to add a Duck.\\
			%Write a method to show (print) the total cost of all statues in the Park.\\
			Write a method that makes all of the ducks in the pond quack one time each.\\
			You may pick anything you like for the string representation of the object.
	
			\begin{flushright}
			\begin{tabular}{|l|}
				\hline
				Pond\\ \hline  	%Class Name
				name\\ ducks\\ \hline		%Instance variables
				add\_duck\\ ducks\_quack \\ \_\_str\_\_ \\ \hline		%methods
			\end{tabular}
			\end{flushright}

		\item
			Create an instance of the Pond class and add two Ducks to it.\\
			Call the method to make all ducks in your pond quack (ducks\_quack).\\
			You can make up any names or colors for Ducks and a Pond.\\
	\end{enumerate}
\pagebreak



%new_question
%%%%%%%%%%%%%%%%%%%%%
	% Problem 5
	% Difficulty: 1
%%%%%%%%%%%%%%%%%%%%%
	\item
	\begin{enumerate}
		\item
			Write a class for a Droid with the below instance variables and methods.\\ 
			The Droid object should have the ability to be passed both initial values.\\  
			You may pick anything you like for the string representation of the object.\\
			A Droid says, "Beep-Bloop-Blop."\\  
			As part of your class, write a method called communicate, that makes a droid communicate!\\
			Hint: for this method, you can just print the words Beep-Bloop-Blop!
			\begin{flushright}
			\begin{tabular}{|l|}
				\hline
				Droid\\ \hline
				designation\\	series\\	 \hline
				get\_series \\ set\_series \\ communicate \\ \_\_str\_\_ \\ \hline
			\end{tabular}
			\end{flushright}

		\item
			Write a class for a Starship. \\
			The class should start (instantiate) with a name, and no Droids in it. \\ 
			Write a method to add a Droid.\\
			Write a method that makes all of the droids in the starship communicate one time each.\\
			You may pick anything you like for the string representation of the object.
	
			\begin{flushright}
			\begin{tabular}{|l|}
				\hline
				Starship\\ \hline  	%Class Name
				name\\ droids\\ \hline		%Instance variables
				add\_droid\\ droids\_communicate \\ \_\_str\_\_ \\ \hline		%methods
			\end{tabular}
			\end{flushright}

		\item
			Create an instance of the Starship class and add two Droids to it.\\
			Call the method to make all droids in your starship communicate (droids\_communicate).\\
			You can make up any designations or series for Droids and a name for a Starship.\\
	\end{enumerate}
\pagebreak


%new_question
%%%%%%%%%%%%%%%%%%%%%
	% Problem 8
	% Difficulty: 1
%%%%%%%%%%%%%%%%%%%%%
	\item 
	\begin{enumerate}
		\item 
			Write a class for a \textit{LLM} with the below instance variables and methods.\\
			A \textit{LLM} should start (be initialized) with both a name and token\_limit (an int).\\
			You may pick anything you like for the string representation of the object.
			\begin{flushright}
			\begin{tabular}{|l|} \hline
				LLM\\ \hline
				name\\ token\_limit\\ \hline
				get\_token\_limit \\ set\_token\_limit \\ \_\_str\_\_ \\ \hline
			\end{tabular}
			\end{flushright}
		
		\item 
			Write a class for \textit{AICompany}. The \textit{AICompany} should start
			with a company\_name and a founding\_year, but with no LLMs. That is, it should
			not have the ability to be initialized with LLMs in it. However, the
			\textit{AICompany} should have the ability to add LLMs as well as display all of the
			LLMs developed by the \textit{AICompany}.\\
			You may pick anything you like for the string representation of the object.
			\begin{flushright}
			\begin{tabular}{|l|} \hline 
				AICompany\\ \hline
				company\_name\\ founding\_year \\ headquarters \\ llms\\ \hline
				get\_headquarters\\ set\_headquarters\\ add\_llm\\ display\_models\\ 
					\_\_str\_\_ \\ \hline
			\end{tabular}
			\end{flushright}

		\item
			Create an instance of the \textit{AICompany} class and add 2 \textit{LLM}s to it.
	\end{enumerate}
\pagebreak



%new_question
%%%%%%%%%%%%%%%%%%%%%
	% Problem 10
	% Difficulty: 1
%%%%%%%%%%%%%%%%%%%%%
	\item
	\begin{enumerate}
		\item
			Write a class for a Book with the below instance variables and methods.\\ 
			The Book object should have the ability to be passed both initial values.\\  
			You may pick anything you like for the string representation of the object.\\
			A Book has an author and can be checked out from a library.\\  
			As part of your class, write a method called display\_info, that displays the book information.\\
			Hint: for this method, you should print the title and author of the book.
			\begin{flushright}
			\begin{tabular}{|l|}
				\hline
				Book\\ \hline
				title \\	author\\	 \hline
				get\_author \\ set\_author \\ display\_info \\ \_\_str\_\_ \\ \hline
			\end{tabular}
			\end{flushright}

		\item
			Write a class for a Library. \\
			The class should start (instantiate) with a library\_name, and no Books in it. \\ 
			Write a method to add a Book.\\
			Write a method that displays all books with their authors.\\
			You may pick anything you like for the string representation of the object.
	
			\begin{flushright}
			\begin{tabular}{|l|}
				\hline
				Library\\ \hline  	%Class Name
				library\_name \\ books\\ \hline		%Instance variables
				add\_book \\ display\_catalog \\ \_\_str\_\_ \\ \hline		%methods
			\end{tabular}
			\end{flushright}

		\item
			Create an instance of the Library class and add two Books to it.\\
			Call the method to display all books in the library (display\_catalog).\\
			You can make up any titles and authors for Books and a library\_name for a Library.\\
	\end{enumerate}
\pagebreak


%new_question

%%%%%%%%%%%%%%%%%%%%%
	% Problem 11
	% Difficulty: 1
%%%%%%%%%%%%%%%%%%%%%
	\item
	\begin{enumerate}
		\item
			Write a class for a Song with the below instance variables and methods.\\ 
			The Song object should have the ability to be passed both initial values.\\  
			You may pick anything you like for the string representation of the object.\\
			A Song has an artist and can be added to a playlist.\\  
			As part of your class, write a method called play, that displays the song information.\\
			Hint: for this method, you should print the title and artist of the song.
			\begin{flushright}
			\begin{tabular}{|l|}
				\hline
				Song\\ \hline
				title \\	artist\\	 \hline
				get\_artist \\ set\_artist \\ play \\ \_\_str\_\_ \\ \hline
			\end{tabular}
			\end{flushright}

		\item
			Write a class for a Playlist. \\
			The class should start (instantiate) with a playlist\_name, and no Songs in it. \\ 
			Write a method to add a Song.\\
			Write a method that plays all songs one after another.\\
			You may pick anything you like for the string representation of the object.
	
			\begin{flushright}
			\begin{tabular}{|l|}
				\hline
				Playlist\\ \hline  	%Class Name
				playlist\_name \\ songs\\ \hline		%Instance variables
				add\_song \\ play\_all \\ \_\_str\_\_ \\ \hline		%methods
			\end{tabular}
			\end{flushright}

		\item
			Create an instance of the Playlist class and add two Songs to it.\\
			Call the method to play all songs in your playlist (play\_all).\\
			You can make up any titles and artists for Songs and a playlist\_name for a Playlist.\\
	\end{enumerate}
\pagebreak


%new_question
%%%%%%%%%%%%%%%%%%%%%
	% Problem 12
	% Difficulty: 1
%%%%%%%%%%%%%%%%%%%%%
	\item
	\begin{enumerate}
		\item
			Write a class for a TVShow with the below instance variables and methods.\\ 
			The TVShow object should have the ability to be passed both initial values.\\  
			You may pick anything you like for the string representation of the object.\\
			A TVShow has a genre and can be added to a dashboard.\\  
			As part of your class, write a method called preview, that displays the show information.\\
			Hint: for this method, you should print the title and genre of the show.
			\begin{flushright}
			\begin{tabular}{|l|}
				\hline
				TVShow\\ \hline
				title \\	genre\\	 \hline
				get\_genre \\ set\_genre \\ preview \\ \_\_str\_\_ \\ \hline
			\end{tabular}
			\end{flushright}

		\item
			Write a class for a NetflixDashboard. \\
			The class should start (instantiate) with a profile\_name, and no TVShows in it. \\ 
			Write a method to add a TVShow.\\
			Write a method that displays all shows saved in the dashboard.\\
			You may pick anything you like for the string representation of the object.
	
			\begin{flushright}
			\begin{tabular}{|l|}
				\hline
				NetflixDashboard\\ \hline  	%Class Name
				profile\_name \\ shows\\ \hline		%Instance variables
				add\_show \\ display\_recommendations \\ \_\_str\_\_ \\ \hline		%methods
			\end{tabular}
			\end{flushright}

		\item
			Create an instance of the NetflixDashboard class and add two TVShows to it.\\
			Call the method to display all shows in your dashboard (display\_recommendations).\\
			You can make up any titles and genres for TVShows and a profile\_name for a NetflixDashboard.\\
	\end{enumerate}
\pagebreak

%end_of_questions
%make sure to leave at least one blank line below






%standard 12.2

%start_of_questions



%new_question
%%%%%%%%%%%%%%%%%%%%%
	% Problem 3
	% Difficulty: 2
%%%%%%%%%%%%%%%%%%%%%
	\item
	\begin{enumerate}
		\item
			Write a class for a Lion with the below instance variables and methods.\\ 
			The Lion object should have the ability to be passed both initial values.\\  
			You may pick anything you like for the string representation of the object.\\
			A Lion says, "Roar."\\  
			As part of your class, write a method called roar, that makes a lion roar!\\
			Hint: for this method, you can just print the word Roar!
			\begin{flushright}
			\begin{tabular}{|l|}
				\hline
				Lion\\ \hline
				name\\	gender\\	 \hline
				get\_name \\ set\_name \\ roar \\ \_\_str\_\_ \\ \hline
			\end{tabular}
			\end{flushright}

		\item
			Write a class for a Zoo. \\
			The class should start (instantiate) with a name, and no Lions in it. \\ 
			Write a method to add a Lion.\\
			Write a method that makes all of the lions in the zoo roar one time each.\\
			Write a method called \textit{count\_lions} which reports the number of male 
			lions and female lions at the zoo. eg. Your code could print: \textit{1 male, 4 female}.\\
			You may pick anything you like for the string representation of the object.
	
			\begin{flushright}
			\begin{tabular}{|l|}
				\hline
				Zoo\\ \hline  	%Class Name
				location\\ lions\\ \hline		%Instance variables
				add\_lion\\ lions\_roar \\ count\_lions \_\_str\_\_ \\ \hline		%methods
			\end{tabular}
			\end{flushright}

		\item
			Create an instance of the Zoo class and add two Lions to it.\\
			Call the method to make all lions in your zoo roar (lions\_roar).\\
			You can make up any names or genders for Lions and a location for a Zoo.\\
	\end{enumerate}
\pagebreak

%new_question
%%%%%%%%%%%%%%%%%%%%%
	% Problem 4
	% Difficulty: 2
%%%%%%%%%%%%%%%%%%%%%
	\item 
	\begin{enumerate}
		\item 
			Write a class for an \textit{Employee} with the below instance variables and methods.\\
			An \textit{Employee} should start (be initialized) with both a name and position.\\
			You may pick anything you like for the string representation of the object.
			\begin{flushright}
			\begin{tabular}{|l|} \hline
				Employee\\ \hline
				name\\ position\\ \hline
				get\_position\\ set\_position\\ \_\_str\_\_ \\ \hline
			\end{tabular}
			\end{flushright}
		
		\item 
			Write a class for a \textit{Department} within a company. The \textit{Department} should 
			start with a name and a budget, but with no employees. That is, it should not have the ability to be
			initialized with employees in it. However, the \textit{Department} should have the ability to add 
			employees as well as show all of the employees in the \textit{Department}.\\
			Additionally, a department is large if it has 10 or more employees. Write a method called 
			\textit{is\_large} which returns True is the department has 10 or more employees and False otherwise.\\
			You may pick anything you like for the string representation of the object.
			\begin{flushright}
			\begin{tabular}{|l|} \hline 
				Department\\ \hline
				dept\_name\\ budget\\ employees\\ \hline
				get\_budget\\ set\_budget\\ add\_employee\\ show\_staff\_list\\ 
					\_\_str\_\_ \\ \hline
			\end{tabular}
			\end{flushright}

		\item
			Create an instance of the \textit{Department} class and add 2 \textit{Employee}s to it.
	\end{enumerate}
\pagebreak


%new_question
%%%%%%%%%%%%%%%%%%%%%
	% Problem 6
	% Difficulty: 2
%%%%%%%%%%%%%%%%%%%%%
	\item
	\begin{enumerate}
		\item
			Write a class for a Post with the below instance variables and methods.\\ 
			The Post object should have the ability to be passed both initial values.\\  
			You may pick anything you like for the string representation of the object.\\
			A Post can be liked with a heart.\\  
			As part of your class, write a method called add\_like, that adds 1 like to the post.\\
				\tab Hint: for this method, you should increase a like counter.\\
			As part of your class, write a method called display, that prints out the post\'s caption.\\			
			\begin{flushright}
			\begin{tabular}{|l|}
				\hline
				Post\\ \hline
				caption\\	likes\\	 \hline
				get\_likes \\ add\_like \\ display \\ \_\_str\_\_ \\ \hline
			\end{tabular}
			\end{flushright}

		\item
			Write a class for a Profile. \\
			The class should start (instantiate) with a username, and no Posts in it. \\ 
			Write a method to add a Post to the user's Profile.\\
			%Write a method that displays all of the posts in the profile one time each.\\
			A post is trending is it has at least 10,000 likes.	 \\Write a method that method 
			that displays all of a user's trending posts one time each.\\ 	
			You may pick anything you like for the string representation of the object.
	
			\begin{flushright}
			\begin{tabular}{|l|}
				\hline
				Profile\\ \hline  	%Class Name
				username\\ posts\\ \hline		%Instance variables
				add\_post\\ display\_trending\_posts \\ \_\_str\_\_ \\ \hline		%methods
			\end{tabular}
			\end{flushright}

		\item
			Create an instance of the Profile class and add two Posts to it.\\
			Call the method to display all the trending posts in the profile (display\_trending\_posts).\\
			You can make up any captions for Posts and a username for a Profile.\\
	\end{enumerate}
\pagebreak


%new_question
%%%%%%%%%%%%%%%%%%%%%
	% Problem 7
	% Difficulty: 2
%%%%%%%%%%%%%%%%%%%%%
	\item
	\begin{enumerate}
		\item
			Write a class for a Product with the below instance variables and methods.\\ 
			The Product object should have the ability to be passed both initial values.\\  
			You may pick anything you like for the string representation of the object.\\
			A Product has a price and can be added to a cart.\\  
			As part of your class, write a method called display\_details, that displays the product information.\\
			Hint: for this method, you should print the name and price of the product.
			\begin{flushright}
			\begin{tabular}{|l|}
				\hline
				Product\\ \hline
				name\\	price\\	 \hline
				get\_price \\ set\_price \\ display\_details \\ \_\_str\_\_ \\ \hline
			\end{tabular}
			\end{flushright}

		\item
			Write a class for a ShoppingCart. \\
			The class should start (instantiate) with a customer\_id, and no Products in it. \\ 
			Write a method to add a Product.\\
			Write a method that calculates the total price of all products in the cart.\\
			You may pick anything you like for the string representation of the object.
	
			\begin{flushright}
			\begin{tabular}{|l|}
				\hline
				ShoppingCart\\ \hline  	%Class Name
				customer\_id \\ products\\ \hline		%Instance variables
				add\_product \\ calculate\_total \\ \_\_str\_\_ \\ \hline		%methods
			\end{tabular}
			\end{flushright}

		\item
			Create an instance of the ShoppingCart class and add two Products to it.\\
			Call the method to calculate the total price of all products in your cart (calculate\_total).\\
			You can make up any names and prices for Products and a customer\_id for a ShoppingCart.\\
	\end{enumerate}
\pagebreak




%new_question
%%%%%%%%%%%%%%%%%%%%%
	% Problem 9
	% Difficulty: 2
%%%%%%%%%%%%%%%%%%%%%
	\item
	\begin{enumerate}
		\item
			Write a class for a MenuItem with the below instance variables and methods.\\ 
			The MenuItem object should have the ability to be passed both initial values.\\  
			You may pick anything you like for the string representation of the object.\\
			A MenuItem has a price and can be added to an order.\\  
			As part of your class, write a method called show\_description, that displays the menu item information.
			Hint: for this method, you should print the name and price of the menu item.
			\begin{flushright}
			\begin{tabular}{|l|}
				\hline
				MenuItem\\ \hline
				name\\	price\\	 \hline
				get\_price \\ set\_price \\ show\_description \\ \_\_str\_\_ \\ \hline
			\end{tabular}
			\end{flushright}

		\item
			Write a class for a Restaurant. \\
			The class should start (instantiate) with a restaurant\_name, and no MenuItems in it. \\ 
			Write a method to add a MenuItem to the restaurant.\\
			Write a method that displays all menu items with their prices.\\
			The lunch menu has all the same MenuItems as the regular menu; however, all MenuItems are \$2 cheaper.
			Write a method called lunch\_menu which displays all MenuItems, but with the prices reduced by \$2. \\	
			You may pick anything you like for the string representation of the object.
	
			\begin{flushright}
			\begin{tabular}{|l|}
				\hline
				Restaurant\\ \hline  	%Class Name
				restaurant\_name \\ menu\_items\\ \hline		%Instance variables
				add\_menu\_item \\ display\_menu \\ \_\_str\_\_ \\ \hline		%methods
			\end{tabular}
			\end{flushright}

		\item
			Create an instance of the Restaurant class and add two MenuItems to it.\\
			Call the method to display all menu items (display\_menu).\\
			You can make up any names and prices for MenuItems and a restaurant\_name for a Restaurant.\\
	\end{enumerate}
\pagebreak

%end_of_questions
%make sure to leave at least one blank line below






\end{enumerate}
\end{document}






%\item Write a program that will convert some amount of pennies into the fewest amount of dollars and coins possible.  For example, 75 pennies is 3 quarters.  86 pennies is 3 quarters, 1 dime, and 1 penny.  130 pennies is 1 dollar, 1 quarter and 1 nickel.  Let the user pick the number of pennies.  You may assume the largest input is 499 pennies.\\
%Hint: the way to do this is to always substitute for the largest denomination if available.  For example, if there is at least 100 pennies substitute for a dollar before quarters.

