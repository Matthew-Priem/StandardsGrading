%standard 12.2

%start_of_questions



%new_question
%%%%%%%%%%%%%%%%%%%%%
	% Problem 3
	% Difficulty: 2
%%%%%%%%%%%%%%%%%%%%%
	\item
	\begin{enumerate}
		\item
			Write a class for a Lion with the below instance variables and methods.\\ 
			The Lion object should have the ability to be passed both initial values.\\  
			You may pick anything you like for the string representation of the object.\\
			A Lion says, "Roar."\\  
			As part of your class, write a method called roar, that makes a lion roar!\\
			Hint: for this method, you can just print the word Roar!
			\begin{flushright}
			\begin{tabular}{|l|}
				\hline
				Lion\\ \hline
				name\\	gender\\	 \hline
				get\_name \\ set\_name \\ roar \\ \_\_str\_\_ \\ \hline
			\end{tabular}
			\end{flushright}

		\item
			Write a class for a Zoo. \\
			The class should start (instantiate) with a name, and no Lions in it. \\ 
			Write a method to add a Lion.\\
			Write a method that makes all of the lions in the zoo roar one time each.\\
			Write a method called \textit{count\_lions} which reports the number of male 
			lions and female lions at the zoo. eg. Your code could print: \textit{1 male, 4 female}.\\
			You may pick anything you like for the string representation of the object.
	
			\begin{flushright}
			\begin{tabular}{|l|}
				\hline
				Zoo\\ \hline  	%Class Name
				location\\ lions\\ \hline		%Instance variables
				add\_lion\\ lions\_roar \\ count\_lions \_\_str\_\_ \\ \hline		%methods
			\end{tabular}
			\end{flushright}

		\item
			Create an instance of the Zoo class and add two Lions to it.\\
			Call the method to make all lions in your zoo roar (lions\_roar).\\
			You can make up any names or genders for Lions and a location for a Zoo.\\
	\end{enumerate}
\pagebreak

%new_question
%%%%%%%%%%%%%%%%%%%%%
	% Problem 4
	% Difficulty: 2
%%%%%%%%%%%%%%%%%%%%%
	\item 
	\begin{enumerate}
		\item 
			Write a class for an \textit{Employee} with the below instance variables and methods.\\
			An \textit{Employee} should start (be initialized) with both a name and position.\\
			You may pick anything you like for the string representation of the object.
			\begin{flushright}
			\begin{tabular}{|l|} \hline
				Employee\\ \hline
				name\\ position\\ \hline
				get\_position\\ set\_position\\ \_\_str\_\_ \\ \hline
			\end{tabular}
			\end{flushright}
		
		\item 
			Write a class for a \textit{Department} within a company. The \textit{Department} should 
			start with a name and a budget, but with no employees. That is, it should not have the ability to be
			initialized with employees in it. However, the \textit{Department} should have the ability to add 
			employees as well as show all of the employees in the \textit{Department}.\\
			Additionally, a department is large if it has 10 or more employees. Write a method called 
			\textit{is\_large} which returns True is the department has 10 or more employees and False otherwise.\\
			You may pick anything you like for the string representation of the object.
			\begin{flushright}
			\begin{tabular}{|l|} \hline 
				Department\\ \hline
				dept\_name\\ budget\\ employees\\ \hline
				get\_budget\\ set\_budget\\ add\_employee\\ show\_staff\_list\\ 
					\_\_str\_\_ \\ \hline
			\end{tabular}
			\end{flushright}

		\item
			Create an instance of the \textit{Department} class and add 2 \textit{Employee}s to it.
	\end{enumerate}
\pagebreak


%new_question
%%%%%%%%%%%%%%%%%%%%%
	% Problem 6
	% Difficulty: 2
%%%%%%%%%%%%%%%%%%%%%
	\item
	\begin{enumerate}
		\item
			Write a class for a Post with the below instance variables and methods.\\ 
			The Post object should have the ability to be passed both initial values.\\  
			You may pick anything you like for the string representation of the object.\\
			A Post can be liked with a heart.\\  
			As part of your class, write a method called add\_like, that adds 1 like to the post.\\
				\tab Hint: for this method, you should increase a like counter.\\
			As part of your class, write a method called display, that prints out the post\'s caption.\\			
			\begin{flushright}
			\begin{tabular}{|l|}
				\hline
				Post\\ \hline
				caption\\	likes\\	 \hline
				get\_likes \\ add\_like \\ display \\ \_\_str\_\_ \\ \hline
			\end{tabular}
			\end{flushright}

		\item
			Write a class for a Profile. \\
			The class should start (instantiate) with a username, and no Posts in it. \\ 
			Write a method to add a Post to the user's Profile.\\
			%Write a method that displays all of the posts in the profile one time each.\\
			A post is trending is it has at least 10,000 likes.	 \\Write a method that method 
			that displays all of a user's trending posts one time each.\\ 	
			You may pick anything you like for the string representation of the object.
	
			\begin{flushright}
			\begin{tabular}{|l|}
				\hline
				Profile\\ \hline  	%Class Name
				username\\ posts\\ \hline		%Instance variables
				add\_post\\ display\_trending\_posts \\ \_\_str\_\_ \\ \hline		%methods
			\end{tabular}
			\end{flushright}

		\item
			Create an instance of the Profile class and add two Posts to it.\\
			Call the method to display all the trending posts in the profile (display\_trending\_posts).\\
			You can make up any captions for Posts and a username for a Profile.\\
	\end{enumerate}
\pagebreak


%new_question
%%%%%%%%%%%%%%%%%%%%%
	% Problem 7
	% Difficulty: 2
%%%%%%%%%%%%%%%%%%%%%
	\item
	\begin{enumerate}
		\item
			Write a class for a Product with the below instance variables and methods.\\ 
			The Product object should have the ability to be passed both initial values.\\  
			You may pick anything you like for the string representation of the object.\\
			A Product has a price and can be added to a cart.\\  
			As part of your class, write a method called display\_details, that displays the product information.\\
			Hint: for this method, you should print the name and price of the product.
			\begin{flushright}
			\begin{tabular}{|l|}
				\hline
				Product\\ \hline
				name\\	price\\	 \hline
				get\_price \\ set\_price \\ display\_details \\ \_\_str\_\_ \\ \hline
			\end{tabular}
			\end{flushright}

		\item
			Write a class for a ShoppingCart. \\
			The class should start (instantiate) with a customer\_id, and no Products in it. \\ 
			Write a method to add a Product.\\
			Write a method that calculates the total price of all products in the cart.\\
			You may pick anything you like for the string representation of the object.
	
			\begin{flushright}
			\begin{tabular}{|l|}
				\hline
				ShoppingCart\\ \hline  	%Class Name
				customer\_id \\ products\\ \hline		%Instance variables
				add\_product \\ calculate\_total \\ \_\_str\_\_ \\ \hline		%methods
			\end{tabular}
			\end{flushright}

		\item
			Create an instance of the ShoppingCart class and add two Products to it.\\
			Call the method to calculate the total price of all products in your cart (calculate\_total).\\
			You can make up any names and prices for Products and a customer\_id for a ShoppingCart.\\
	\end{enumerate}
\pagebreak




%new_question
%%%%%%%%%%%%%%%%%%%%%
	% Problem 9
	% Difficulty: 2
%%%%%%%%%%%%%%%%%%%%%
	\item
	\begin{enumerate}
		\item
			Write a class for a MenuItem with the below instance variables and methods.\\ 
			The MenuItem object should have the ability to be passed both initial values.\\  
			You may pick anything you like for the string representation of the object.\\
			A MenuItem has a price and can be added to an order.\\  
			As part of your class, write a method called show\_description, that displays the menu item information.
			Hint: for this method, you should print the name and price of the menu item.
			\begin{flushright}
			\begin{tabular}{|l|}
				\hline
				MenuItem\\ \hline
				name\\	price\\	 \hline
				get\_price \\ set\_price \\ show\_description \\ \_\_str\_\_ \\ \hline
			\end{tabular}
			\end{flushright}

		\item
			Write a class for a Restaurant. \\
			The class should start (instantiate) with a restaurant\_name, and no MenuItems in it. \\ 
			Write a method to add a MenuItem to the restaurant.\\
			Write a method that displays all menu items with their prices.\\
			The lunch menu has all the same MenuItems as the regular menu; however, all MenuItems are \$2 cheaper.
			Write a method called lunch\_menu which displays all MenuItems, but with the prices reduced by \$2. \\	
			You may pick anything you like for the string representation of the object.
	
			\begin{flushright}
			\begin{tabular}{|l|}
				\hline
				Restaurant\\ \hline  	%Class Name
				restaurant\_name \\ menu\_items\\ \hline		%Instance variables
				add\_menu\_item \\ display\_menu \\ \_\_str\_\_ \\ \hline		%methods
			\end{tabular}
			\end{flushright}

		\item
			Create an instance of the Restaurant class and add two MenuItems to it.\\
			Call the method to display all menu items (display\_menu).\\
			You can make up any names and prices for MenuItems and a restaurant\_name for a Restaurant.\\
	\end{enumerate}
\pagebreak

%end_of_questions
%make sure to leave at least one blank line below



