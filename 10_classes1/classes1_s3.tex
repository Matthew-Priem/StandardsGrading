%standard 11.3

%start_of_questions



%new_question

%%%%%%%%%%%%%%%%%%%%%
	% Problem 5
	% Difficulty: 3
%%%%%%%%%%%%%%%%%%%%%
	\item Create an \textit{Employee} class.\\
	\begin{minipage}{.6\textwidth}		
		An \textit{Employee} has
		\begin{itemize}
			\item A name
			\item A title
			\item A salary	
		\end{itemize}
	
		An \textit{Employee} can do
		\begin{itemize}
			\item a greeting
			\item request raise
		\end{itemize}
	\end{minipage} 
	%
	\begin{minipage}{.4\textwidth}
		This class ``looks'' like 
				
		\vspace*{1em}
		\begin{tabular}{|l|}
			\hline Employee\\ \hline
			name\\ title\\ salary\\ \hline
			greeting\\ request\_raise \\  \hline
		\end{tabular}
	\end{minipage}

	\vspace*{2ex}
	You should write getters and setters for each of the instance variables.\\

	A greeting should be of the form: \underline{Hello.  My name is \textit{name}.  
	I'm the \textit{title}.}\\
	\tab \tab eg. Hello.  My name is Eugene.  I'm the CEO.\\

	A raise request should request a \underline{6\%} raise.\\  It should be of the form: 
	I'm currently making \textit{salary}.  I'd like new salary of \textit{new amount}.\\
	\tab \tab eg. I'm currently making \$100.  I'd like new salary of \$106.\\



%new_question

%%%%%%%%%%%%%%%%%%%%%
	% Problem 6
	% Difficulty: 3
%%%%%%%%%%%%%%%%%%%%%
	\item Create a \textit{Student} class.\\
	\begin{minipage}{.6\textwidth}		
		A \textit{Student} has
		\begin{itemize}
			\item A name
			\item A major
			\item A GPA	
		\end{itemize}
	
		A \textit{Student} can do
		\begin{itemize}
			\item introduce themselves
			\item study for exam
		\end{itemize}
	\end{minipage} 
	%
	\begin{minipage}{.4\textwidth}
		This class \csq{looks} like
		 
		\vspace*{1em}
		\begin{tabular}{|l|}
			\hline Student\\ \hline
			name\\ major\\ GPA\\ \hline
			introduce\\ study\_for\_exam \\  \hline
		\end{tabular}
	\end{minipage}

	\vspace*{2ex}
	You should write getters and setters for each of the instance variables.\

	An introduction should be of the form: \underline{Hi, I'm  \textit{name}.  
	I'm studying \textit{major}.}\\
	\tab \tab eg. Hi. I'm Maria. I'm studying Computer Science.\\

	Studying for an exam should increase the GPA by \underline{0.2} points. (up to a maximum of 4.0)\\  
	It should be of the form: \\
	I'm hitting the books! My GPA increased from \textit{old GPA} to \textit{new GPA}.\\
	\tab \tab eg. I'm hitting the books! My GPA increased from 3.5 to 3.7.\\




%new_question

%%%%%%%%%%%%%%%%%%%%%
	% Problem 10
	% Difficulty: 3
%%%%%%%%%%%%%%%%%%%%%
	\item Create a \textit{Vector} class.\\
	\begin{minipage}{.6\textwidth}
		A \textit{Vector} has
		\begin{itemize}
			\item x\_direction 
			\item y\_direction
		\end{itemize}
		
		A \textit{Vector} can do
		\begin{itemize}
			\item get\_magnitude
		\end{itemize}
	\end{minipage}
		%
	\begin{minipage}{.4\textwidth}
		This class ``looks'' like 
				
		\vspace*{1em}
		\begin{tabular}{|l|}
			\hline Vector\\ \hline
			x\_direction\\ y\_direction\\ \hline
			get\_magnitude\\  \hline
		\end{tabular}
	\end{minipage}

	\vspace*{2ex}
	Create a constructor method that initializes all instance variables.\\
	You should write getters and setters for each of the instance variables.\\
	Instantiate an instance of the class. You may pass any initial values of your choosing.
	
	Hint: magnitude is calculated as $\sqrt{x^2 + y^2}$.


%new_question

%%%%%%%%%%%%%%%%%%%%%
	% Problem 11
	% Difficulty: 3
%%%%%%%%%%%%%%%%%%%%%
	\item Create a \textit{ColorRGB} class.\\
	\begin{minipage}{.6\textwidth}		
		A \textit{ColorRGB} has
		\begin{itemize}
			\item red 
			\item green
			\item blue
		\end{itemize}

		A \textit{ColorRGB} can do
		\begin{itemize}
			\item to\_grayscale
		\end{itemize}
	\end{minipage}
		%
	\begin{minipage}{.4\textwidth}
		This class ``looks'' like 
				
		\vspace*{1em}
		\begin{tabular}{|l|}
			\hline ColorRGB\\ \hline
			red\\ green\\ blue\\ \hline
			to\_grayscale\\  \hline
		\end{tabular}
	\end{minipage}

	\vspace*{2ex}
	Create a constructor method that initializes all instance variables.\\
	You should write getters and setters for each of the instance variables.\\
	Instantiate an instance of the class. You may pass any initial values of your choosing.

	The to\_grayscale() method should return the grayscale value calculated as: 
		$$0.3 * \text{red} + 0.59 * \text{green} + 0.11 * \text{blue}$$
	That is, it will just return a number (a float).

	%Create an \_\_str\_\_ method that returns a string in the format:\\
	%"RGB: ([red], [green], [blue])"\\
	%\tab \tab eg. "RGB: (255, 128, 64)".\\


%new_question

%%%%%%%%%%%%%%%%%%%%%
	% Problem 12
	% Difficulty: 3
%%%%%%%%%%%%%%%%%%%%%
	\item Create a \textit{TemperatureInCelsius} class.\\
	\begin{minipage}{.6\textwidth}		
		A \textit{TemperatureInCelsius} has
		\begin{itemize}
			\item temp\_value
		\end{itemize}

		A \textit{TemperatureInCelsius} can do
		\begin{itemize}
			\item to\_fahrenheit
		\end{itemize}
	\end{minipage}
		%
	\begin{minipage}{.4\textwidth}
		This class ``looks'' like 
				
		\vspace*{1em}
		\begin{tabular}{|l|}
			\hline TemperatureInCelsius\\ \hline
			temp\_value\\ \ \\  \hline
			to\_fahrenheit\\ \ \\ \hline
		\end{tabular}
	\end{minipage}



	\vspace*{2ex}
	Clarification: temp\_value is the temperature in Celsius.\\
	Create a constructor method that initializes all instance variables.\\
	You should write getters and setters for each of the instance variables.\\
	Instantiate an instance of the class. You may pass any initial values of your choosing.
	
	The to\_fahrenheit() method should return the temperature in Fahrenheit calculated as:\\
	Fahrenheit = (Celsius * 9/5) + 32.\

	%Create an \_\_str\_\_ method that returns a string in the format:\\
	%"Temp: [celsius]C, Humidity: [humidity]\%, Pressure: [pressure]hPa"\\
	%\tab \tab eg. "Temp: 25.0C, Humidity: 65\%, Pressure: 1013hPa".\\


%new_question

%%%%%%%%%%%%%%%%%%%%%
	% Problem 13
	% Difficulty: 3
%%%%%%%%%%%%%%%%%%%%%
	\item Create a \textit{Rectangle} class.\\
	\begin{minipage}{.6\textwidth}
		A \textit{Rectangle} has
		\begin{itemize}
			\item width 
			\item height
		\end{itemize}

		A \textit{Rectangle} can do
		\begin{itemize}
			\item calculate\_area
		\end{itemize}
	\end{minipage}
		%
	\begin{minipage}{.4\textwidth}
		This class ``looks'' like 
				
		\vspace*{1em}
		\begin{tabular}{|l|}
			\hline Rectangle\\ \hline
			width\\ height \\  \hline
			calculate\_area\\ \hline
		\end{tabular}
	\end{minipage}


	\vspace*{2ex}
	Create a constructor method that initializes all instance variables.\\
	You should write getters and setters for each of the instance variables.\\
	Instantiate an instance of the class. You may pass any initial values of your choosing.
	
	The calculate\_area() method should return the area calculated as: width * height.\\

	%Create an \_\_str\_\_ method that returns a string in the format:\\
	%"Rectangle([width] × [height], [color])"\\
	%\tab \tab eg. "Rectangle(10.5 × 20.0, blue)".\\


%new_question

%%%%%%%%%%%%%%%%%%%%%
	% Problem 14
	% Difficulty: 3
%%%%%%%%%%%%%%%%%%%%%
	\item Create a \textit{Circle} class.\\	
	\begin{minipage}{.6\textwidth}
		A \textit{Circle} has
		\begin{itemize}
			\item radius 
		\end{itemize}

		A \textit{Circle} can do
		\begin{itemize}
			%\item calculate\_area()
			\item calculate\_circumference
		\end{itemize}
	\end{minipage}
		%
	\begin{minipage}{.4\textwidth}
		This class ``looks'' like 
				
		\vspace*{1em}
		\begin{tabular}{|l|}
			\hline Circle\\ \hline
			radius\\ \ \\  \hline
			calculate\_circumference\\ \hline
		\end{tabular}
	\end{minipage}

	\vspace*{2ex}
	Create a constructor method that initializes all instance variables.\\
	You should write getters and setters for each of the instance variables.\\
	Instantiate an instance of the class. You may pass any initial values of your choosing.	

	%The calculate\_area() method should return the area calculated as: $\pi \cdot \text{radius}^2$.\\
	The calculate\_circumference() method should return the circumference calculated as: $2 \cdot \pi \cdot \text{radius}$.\\

	%Create an \_\_str\_\_ method that returns a string in the format:\\
	%"Circle(radius=[radius], color=[color], filled=[filled])"\\
	%\tab \tab eg. "Circle(radius=5.0, color=red, filled=True)".\\


%end_of_questions
%make sure to leave at least one blank line below


