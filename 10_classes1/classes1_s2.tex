%standard 11.2


%start_of_questions


%new_question

%%%%%%%%%%%%%%%%%%%%%
	% Problem 7
	% Difficulty: 2
%%%%%%%%%%%%%%%%%%%%%
	\item Create a \textit{Vehicle} class.\\
	\begin{minipage}{.6\textwidth}	
		A \textit{Vehicle} has
		\begin{itemize}
			\item make 
			\item model
			\item year	
		\end{itemize}
		
		A \textit{Vehicle} can do
		\begin{itemize}
			\item \textit{print\_vehicle\_type}
		\end{itemize}
	\end{minipage}
	%
	\begin{minipage}{.4\textwidth}
		This class ``looks'' like 
				
		\vspace*{1em}
		\begin{tabular}{|l|}
			\hline Vehicle\\ \hline
			make\\ model\\ year\\ \hline
			print\_vehicle\_type \\  \hline
		\end{tabular}
	\end{minipage}

	\vspace*{2ex}
	Create a constructor method that initializes all instance variables.\\
	You should write getters and setters for each of the instance variables.\\
	Instantiate an instance of the class. You may pass any initial values of your choosing.

	Write a method called \textit{print\_vehicle\_type}, which prints in the form ``[year] [make] [model]''\\
	example. ``2021 Toyota Camry''.\\

	%Create a \_\_str\_\_ method that returns a string in the format:\\
	%"[year] [make] [model]"\\
	%\tab \tab eg. "2021 Toyota Camry".\\


%new_question

%%%%%%%%%%%%%%%%%%%%%
	% Problem 8
	% Difficulty: 2
%%%%%%%%%%%%%%%%%%%%%
	\item Create a \textit{Course} class.\\
	\begin{minipage}{.6\textwidth}
		A \textit{Course} has
		\begin{itemize}
			\item course\_code 
			\item course\_name
			\item instructor	
		\end{itemize}

		An \textit{Course} can do
		\begin{itemize}
			\item \textit{print\_info}
		\end{itemize}	
	\end{minipage}
	%
	\begin{minipage}{.4\textwidth}
		This class ``looks'' like 
				
		\vspace*{1em}
		\begin{tabular}{|l|}
			\hline Course\\ \hline
			course\_code\\ course\_name\\ instructor\\ \hline
			print\_info\\  \hline
		\end{tabular}
	\end{minipage}

	\vspace*{2ex}
	Create a constructor method that initializes all instance variables.\\
	You should write getters and setters for each of the instance variables.\\
	Instantiate an instance of the class. You may pass any initial values of your choosing.

	Write a method called \textit{print\_info}, which prints in the form \\
		\tab \tab \tab ``[course\_code]: [course\_name] taught by [instructor]''\\
	example. ``CIS101: Introduction to programming taught by Matt''.\\

	%Create a \_\_str\_\_ method that returns a string in the format:\\
	%"[course\_code]: [course\_name] taught by [instructor], has a max capacity of [max\_capacity]"\\
	%\tab \tab eg. "CIS101: Introduction to programming taught by Dr.Smith, has a max capacity of 25".\\


%new_question

%%%%%%%%%%%%%%%%%%%%%
	% Problem 9
	% Difficulty: 2
%%%%%%%%%%%%%%%%%%%%%
	\item Create a \textit{Point} class.\\
	\begin{minipage}{.6\textwidth}		
		A \textit{Point} has
		\begin{itemize}
			\item x\_coordinate 
			\item y\_coordinate 
		\end{itemize}

		A \textit{Point} can do
		\begin{itemize}
			\item \textit{print\_info}
		\end{itemize}
	\end{minipage}
	%
	\begin{minipage}{.4\textwidth}
		This class ``looks'' like 
				
		\vspace*{1em}
		\begin{tabular}{|l|}
			\hline Course\\ \hline
			x\_coordinate\\ y\_coordinate\\ \hline
			print\_info\\  \hline
		\end{tabular}
	\end{minipage}

	\vspace*{2ex}
	Create a constructor method that initializes all instance variables.\\
	You should write getters and setters for each of the instance variables.\\
	Instantiate an instance of the class. You may pass any initial values of your choosing.

	Write a method called \textit{print\_info}, which prints in the form \\
		\tab \tab \tab ``(x,y)=([x\_coordinate], [y\_coordinate])''\\
	example. ``(x,y)=( 4, 5 )''.\\

	%Create a \_\_str\_\_ method that returns a string in the format:\\
	%"( [x\_coordinate], [y\_coordinate], [z\_coordinate] )"\\
	%\tab \tab eg. "( 4, 5, 6 )".\\


%end_of_questions
%make sure to leave at least one blank line below


